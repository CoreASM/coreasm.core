% !TeX spellcheck = en_US
% !TeX program = pdflatex 
% ---------------------------------------------------------
% CoreASM-UserManual.tex 	$Revision: #1 $
%
% "CoreASM User Manual"
%
% Copyright (c) 2006-2007 Roozbeh Farahbod
%
% This work is licensed under the Creative Commons 
% Attribution-NonCommercial-NoDerivs License. To view 
% a copy of this license, visit the following link:
%
%   http://creativecommons.org/licenses/by-nc-nd/2.0/ca/ 
% 
% ---------------------------------------------------------

\documentclass{article}

%\DeclareFontShape{OT1}{cmtt}{bx}{n}{
%  <5><6><7><8><9><10><10.95><12><14.4><17.28><20.74><24.88>cmttb10}{}


\newcommand{\version}{1.6.2-beta}

\usepackage{amsmath}
\usepackage{graphicx}
\usepackage{color}
\usepackage{amsfonts}
\usepackage{amssymb}
\usepackage{url}
\usepackage{stmaryrd}
\usepackage{fancyhdr}
\usepackage{float}
\usepackage{makeidx}
\usepackage{psboxit}
\usepackage{xspace}
\usepackage{bold-extra}
\usepackage{cmtt}
\definecolor{darkgreen}{rgb}{0,0.3,0}
\usepackage[colorlinks=true, citecolor=blue, linkcolor=blue, urlcolor=darkgreen]{hyperref}

\setlength{\oddsidemargin}{0.2in}
\setlength{\textwidth}{5.8in}
\addtolength{\parskip}{0.07in}

% ---------------------------------------------------------
% asmcommands.tex  (last modified: $Date: 2009-07-30 15:18:35 +0200 (Thu, 30 Jul 2009) $)
%
% A series of LaTeX commands defined for 
% typesetting ASM specifications
%
% Copyright (c) 2005-2009 Roozbeh Farahbod
% Copyright (c) 2005 Uwe Glaesser
% Copyright (c) 2005 Mona Vajihollahi
% Copyright (c) 2005 Vincenzo Gervasi
% Copyright (c) 2005 Mashaal Memon
%
% This work is licensed under the Creative Commons 
% Attribution-NonCommercial-NoDerivs License. To view 
% a copy of this license, visit the following link:
%
%   http://creativecommons.org/licenses/by-nc-nd/2.0/ca/ 
% 
% ---------------------------------------------------------

\newcommand{\CoreASM}{{\sffamily CoreASM}\xspace}
\newcommand{\JASMine}{{\sffamily JASMine}\xspace}
\newcommand{\Carma}{{\sffamily Carma}\xspace}

% ----------------- ASM Rule Environments
\newcommand{\bmRule}{ \noindent \small $ \begin{array}{l} }
\newcommand{\emRule}{ \end{array} $  \normalsize }

% ASM rule block, begin and end
\newcommand{\bRule}{ \noindent \small \begin{math} \begin{array}{l} }
\newcommand{\eRule}{ \end{array}\end{math}  \normalsize }

% ASM rule block with indention, begin and end
\newcommand{\biRule}{ \vspace{2mm}\small \begin{math} \begin{array}{l} }
\newcommand{\eiRule}{ \end{array}\end{math}  \normalsize \vspace{2mm} }


% Fancy ASM rule block, with line and title, begin and end
\newcommand{\blRule}[1]{\noindent\begin{minipage}{\linewidth}\vspace{3mm}\small\rule{\linewidth}{0.3mm}\\\mbox{~}\hfill\raisebox{4pt}[0pt]{\sf\scriptsize #1}\vspace{-2pt}\\\begin{math}\begin{array}{l}}

\newcommand{\elRule}{\end{array}\end{math}\vspace{1mm}\\\rule[2mm]{\linewidth}{0.15mm}\vspace{1mm}\end{minipage}\normalsize}

% Hold and resume fancy ASM Rule block
\newcommand{\holdLRule}{\end{array}\end{math}\end{minipage}\normalsize}
\newcommand{\resumeLRule}{\noindent\begin{minipage}{\linewidth}\vspace{3mm}\small\begin{math}\begin{array}{l}}

\newcommand{\bpRule}{ \noindent \small $ \begin{array}{l} \rule{0.94\textwidth}{0mm}  \\ \hspace{0.02\textwidth} \begin{array}{l} }
\newcommand{\epRule}{ \end{array} \\ \rule[2mm]{0.94\textwidth}{0mm} \end{array} $ \normalsize }

\newcommand{\bcRule}{ \small \begin{displaymath}  \noindent  \begin{array}{l} }
\newcommand{\ecRule}{ \end{array}\end{displaymath} \normalsize }


% ----------------- Keywords 
\newcommand{\Rkeyword}[1]{\mbox{\bf #1}~}
\newcommand{\Rforall}{\mbox{\bf forall }}
\newcommand{\Rforeach}{\mbox{\bf foreach }}
\newcommand{\Rcase}{\mbox{\bf case }}
\newcommand{\casedivider}{\vspace{2mm}\\}
\newcommand{\Rendcase}{\mbox{\bf endcase }}
\newcommand{\Rskip}{\mbox{\bf skip}\xspace}
\newcommand{\Rpar}{\mbox{\bf~par~}}
\newcommand{\Rseqblock}{\mbox{\bf seqblock}\xspace}
\newcommand{\Rendseqblock}{\mbox{\bf endseqblock}\xspace}
\newcommand{\Rseq}{\mbox{\bf~seq~}}
\newcommand{\Rnext}{\mbox{\bf~next~}}
\newcommand{\Rof}{\mbox{\bf ~of}}
\newcommand{\Rnew}{\mbox{\bf new }}
%\newcommand{\Rprint}{\mbox{\bf print~}}
\newcommand{\Ron}{\mbox{\bf onsignal }}
\newcommand{\Rtrigger}{\mbox{\bf trigger }}
\newcommand{\Rchoose}{\mbox{\bf choose }}
\newcommand{\Rextend}{\mbox{\bf extend }}
\newcommand{\Rlet}{\mbox{\bf let}~}
\newcommand{\Rwhere}{\Rroutine{where}~}
\newcommand{\Rif}{\mbox{\bf if}~}
\newcommand{\Rin}{~\mbox{\bf in}~}
\newcommand{\Ror}{~\mbox{\bf or}~}
\newcommand{\Rand}{~\mbox{\bf and}~}
\newcommand{\Rnot}{~\mbox{\bf not}~}
\newcommand{\Rthen}{~\mbox{\bf then}~}
\newcommand{\Relse}{\mbox{\bf else}~}
\newcommand{\Relseif}{\mbox{\bf else~if}~}
\newcommand{\Rwith}{~\mbox{\bf with}~}
\newcommand{\Rholds}{~\mbox{\bf holds}~}
\newcommand{\Rstop}{\mbox{\bf stop}~}
\newcommand{\Radd}{\mbox{\bf add}~}
\newcommand{\Rto}{~\mbox{\bf to}~}
\newcommand{\Rremove}{\mbox{\bf remove}~}
\newcommand{\Rfrom}{~\mbox{\bf from}~}
\newcommand{\Rself}{\mbox{\it self}\,}
\newcommand{\Rundef}{\mbox{\it undef}\xspace}
\newcommand{\Rtrue}{\mbox{\it true}\xspace}
\newcommand{\Rfalse}{\mbox{\it false}\xspace}
\newcommand{\Rotherwise}{\mbox{\bf otherwise }}
\newcommand{\Rimport}{\mbox{\bf import }}
\newcommand{\Rrule}{\mbox{\bf rule }}
\newcommand{\Rreturn}{\mbox{\bf return }}
\newcommand{\Rreturnlocal}{\mbox{\bf return local }}
\newcommand{\Rresult}{\mbox{\bf result\xspace}}
\newcommand{\Rresultarrow}{\mbox{\bf $\leftarrow$ }}
\newcommand{\Riterate}{\mbox{\bf iterate }}
\newcommand{\Rwhile}{\mbox{\bf while }}
\newcommand{\Rdo}{\mbox{\bf ~do}\xspace}
\newcommand{\Rcall}{\mbox{\bf call~}}
\newcommand{\Rlocal}{\mbox{\bf local~}}
\newcommand{\Rtry}{\mbox{\bf try~}}
\newcommand{\Rcatch}{\mbox{\bf~catch~}}
\newcommand{\Rabstract}{\mbox{\bf abstract~}}
\newcommand{\Rnative}{\mbox{\bf native~}}
\newcommand{\Rifnone}{\mbox{\bf ifnone}~}
\newcommand{\Rnthen}{\mbox{\bf then}~}
\newcommand{\Rnelse}{\mbox{\bf else}~}
\newcommand{\Rkwhere}{~\mbox{\bf where}~}
\newcommand{\Ris}{~\mbox{\bf is}~}
\newcommand{\Rexistential}{\mbox{\bf exists\xspace}}
\newcommand{\Runiversal}{\mbox{\bf forall\xspace}}
\newcommand{\Rmemberof}{~\mbox{\bf memberof}~}

\newcommand{\Rmonitored}{\mbox{\bf \small monitored}\xspace}
%\newcommand{\Rcontrolled}{\mbox{\bf controlled}}
\newcommand{\Rcontrolled}{}
\newcommand{\Rderived}{\mbox{\footnotesize \bf derived}\xspace}
\newcommand{\Rstatic}{\mbox{\bf static}}
\newcommand{\Rout}{\mbox{\bf out}}


% ----------------- Styles
\newcommand{\Rfun}[1]{\mbox{\it #1}}
\newcommand{\Rdom}[1]{\ensuremath{\mbox{\scshape\small #1}}\xspace}
\newcommand{\Rcom}[1]{\mbox{\textcolor[gray]{0.5}{{\scriptsize //} #1}}}
\newcommand{\Rtcom}[1]{\Rcom{\rule[.8mm]{0.2\textwidth}{0.1mm} #1 \rule[.8mm]{0.2\textwidth}{0.1mm}}} 
\newcommand{\Rroutine}[1]{\textsf{#1}}
\newcommand{\RroutineHeader}[1]{\textsf{\textbf{#1}}\rindex{#1}}
\newcommand{\Rdomain}[1]{\mbox{\bf domain}~\Rdom{#1}}
\newcommand{\Rqterm}[1]{\mbox{\small \sf ``#1''}}
%\newcommand{\Rqterm}[1]{{\raisebox{1pt}{\tiny$\ll$}}\mbox{\sf #1}\raisebox{1pt}{\tiny$\gg$}}
\newcommand{\sRfun}[1]{\mbox{\scriptsize\it #1}}
\newcommand{\Rfunsub}[2]{\ensuremath{\Rfun{#1}_{\mbox{\scriptsize\it #2}}}}
\newcommand{\Rfsig}[3]{\Rfun{#1}\findex{#1}: \Rdom{#2}\mapsto\Rdom{#3}}
\newcommand{\Rnfsig}[2]{\Rfun{#1}\findex{#1}: \Rdom{#2}}
\newcommand{\Rmapping}[2]{\ensuremath{\Rdom{#1}\mapsto\Rdom{#2}}}
\newcommand{\Rdomintro}[1]{\dindex{#1}\Rdom{#1}}

\newcommand{\Nat}{{\mathbb N}}
\newcommand{\seq}[1]{\ensuremath{\langle{#1}\rangle}}
\newcommand{\findex}[1]{\index{#1@{\emph{#1}}}}
\newcommand{\fsubindex}[2]{\index{#1@{\emph{#1}}!#2}}
\newcommand{\dindex}[1]{\index{#1@{\scshape #1}}}
\newcommand{\rindex}[1]{\index{#1@\Troutine{#1}}}
\newcommand{\patindex}[1]{\index{#1 pattern@{\bf #1} pattern}}
\newcommand{\oprindex}[1]{\index{#1 operator@{\bf #1} operator}}

\newcommand{\oprline}[1]{\vspace{5mm}\noindent $\bullet$~{\bf #1}}


% ----------------- Styles in Text
\newcommand{\Troutine}[1]{\textsf{#1}}
\newcommand{\Tfun}[1]{\textsf{\small \it #1}}
\newcommand{\Tdom}[1]{\textsf{\small #1}}
\newcommand{\Tfsig}[3]{\ensuremath{\Rfun{#1}\findex{#1}: \Rdom{#2}\mapsto\Rdom{#3}}}
\newcommand{\code}[1]{{\ttfamily #1}}


% ----------------- Miscellaneous
\newcommand{\Rcup}{~\cup~}
\newcommand{\Rfname}[1]{\Rfun{#1Function}}


% ---------------------------------------------------------
% misc-commands.tex     1.5     19-Aug-2005
%
% Miscellaneous LaTeX Commands for
% "Design and Specification of the
%  CoreASM Execution Engine"
%
% Copyright (c) 2005 Roozbeh Farahbod
% Copyright (c) 2005 Vincenzo Gervasi
% Copyright (c) 2005 Mashaal Memon
%
% ---------------------------------------------------------

% ----------------- Universe Names
\newcommand{\Node}{\Rdom{Node}}
\newcommand{\Var}{\String}
\newcommand{\Updates}{\Rdom{Updates}}
\newcommand{\Upd}{\Updates}
\newcommand{\Loc}{\Rdom{Loc}}
\newcommand{\Back}{\Rdom{Background}}
\newcommand{\Class}{\Rdom{Class}}
\newcommand{\Token}{\Rdom{Token}}
\newcommand{\Pattern}{\Rdom{Pattern}}
\newcommand{\Element}{\Rdom{Element}}
\newcommand{\Number}{\Rdom{Number}}
\newcommand{\element}{Element}
\newcommand{\Val}{\Element}
\newcommand{\UniverseElement}{\Rdom{UniverseElement}}
\newcommand{\Operation}{\Rdom{Operation}}
\newcommand{\Property}{\String}
\newcommand{\PropertyValue}{\String}
\newcommand{\FunctionElement}{\Rdom{FunctionElement}}
\newcommand{\Signature}{\Rdom{Signature}}
\newcommand{\MathFuncElement}{\Rdom{MathFunctionElement}}
\newcommand{\FClass}{\Rdom{FuncClass}}
\newcommand{\Rule}{\Rdom{Rule}}
\newcommand{\RuleElement}{\Rdom{Rule}}
\newcommand{\IdentifierElement}{\Rdom{IdentifierElement}}
\newcommand{\Plugin}{\Rdom{Plugin}}
\newcommand{\SuperUnivElement}{\Rdom{SuperUniverseElement}}
\newcommand{\SuperUnivName}{\ensuremath{\Rqterm{SUPER\_UNIVERSE}}}
\newcommand{\Action}{\Rdom{Action}}
\newcommand{\Storage}{\Rdom{Storage}}
\newcommand{\Engine}{\Rdom{Engine}}
\newcommand{\CAPI}{\Rdom{ControlAPI}}
\newcommand{\Spec}{\Rdom{Spec}}
\newcommand{\EngineMode}{\Rdom{EngineMode}}
\newcommand{\State}{\Rdom{State}}
\newcommand{\Boolean}{\Rdom{Boolean}}
%\newcommand{\BooleanElement}{\Rdom{BooleanElement}}
\newcommand{\BooleanElement}{\Rdom{BooleanElement}}
\newcommand{\Int}{\Rdom{Integer}}
\newcommand{\Real}{\Rdom{Real}}
%\newcommand{\IntElement}{\Rdom{IntElement}}
\newcommand{\NumberElement}{\Rdom{NumberElement}}
\newcommand{\IntRangeElement}{\Rdom{NumberRange}}
%\newcommand{\String}{\Rdom{String}}
\newcommand{\String}{\Rdom{String}}
\newcommand{\Agent}{\Rdom{Agent}}
\newcommand{\Program}{\Rdom{Program}}
\newcommand{\Observer}{\Rdom{EventObserver}}
\newcommand{\OprImp}{\Rdom{OperatorImp}}
\newcommand{\Operator}{\Rdom{Operator}}
\newcommand{\PluginAction}{\Rdom{PluginAction}}
\newcommand{\SetElement}{\Rdom{SetElement}}
\newcommand{\Set}{\Rdom{Set}}
\newcommand{\Flag}{\Rdom{Flag}}
\newcommand{\Name}{\Rdom{Name}}
\newcommand{\NamedElement}{\Rdom{NamedElement}}
\newcommand{\EnumerationBackgroundElement}{\Rdom{EnumerationBackground}}

% ----------------- Quoted Names
\newcommand{\epTypeChecking}{\Rqterm{TypeChecking}}
\newcommand{\epvIgnore}{\Rqterm{Ignore}}
\newcommand{\epvWarning}{\Rqterm{Warning}}
\newcommand{\epvError}{\Rqterm{Error}}


% ----------------- Functions and Rules
\newcommand{\token}[1]{\mbox{\sf #1}}
\newcommand{\pos}{\ensuremath{\Rfun{pos}}\xspace}
\newcommand{\vpos}{\ensuremath{\Rfun{value}(\Rfun{pos})}}
\newcommand{\upos}{\ensuremath{\Rfun{updates}(\Rfun{pos})}}
\newcommand{\lpos}{\ensuremath{\Rfun{loc}(\Rfun{pos})}}
\newcommand{\TT}{\mbox{\sf tt}}
\newcommand{\FF}{\mbox{\sf ff}}
\newcommand{\UU}{\mbox{\sf uu}}
\newcommand{\RESULT}{\mbox{\bf result}}
\newcommand{\ERROR}[1]{\Rroutine{Error}(\mbox{`#1'})}
\newcommand{\WARN}[1]{\Rroutine{Warning}(\mbox{`#1'})}
\newcommand{\Bool}{\ensuremath{\mathbb B}}
\newcommand{\sbkg}[1]{\mbox{\scriptsize\it bkg}{\scriptsize (#1)}}
\newcommand{\push}{\ensuremath{\Rroutine{PushState}}\xspace}
\newcommand{\pop}{\ensuremath{\Rroutine{PopState}}\xspace}
\newcommand{\diff}{\ensuremath{\Rroutine{Diff}}\xspace}
\newcommand{\apply}[1]{\ensuremath{\Rroutine{Apply}({#1})}\xspace}
\newcommand{\clearTree}{\ensuremath{\Rroutine{ClearTree}}\xspace}
\newcommand{\getValue}[1]{\Rfun{getValue}(#1)}
\newcommand{\engineMode}{\ensuremath{\Rfun{engineMode}}}
\renewcommand{\exp}{\ensuremath{\Rfun{exp}}}
\newcommand{\id}{\ensuremath{\Rfun{x}\xspace}}
%\newcommand{\uneval}{\raisebox{0.7ex}{\framebox{\hspace{1ex}e}}}
%\newcommand{\unevale}{\raisebox{0.7ex}{\framebox[0.7em]{}}\hspace{-0.4em}\textcolor[rgb]{0.5, 0.5, 0.5}{e}}
%\newcommand{\node}{\rule{0.7em}{1.5ex}}

\newlength{\circwidth}
\newcommand{\gencircled}[2]{%
\settowidth{\circwidth}{\makebox{#2}}%
\hspace*{.5\circwidth}%
\makebox[0pt]{#2}%
\makebox[0pt]{\small{#1}}%
\hspace*{.5\circwidth}%
}
\newcommand{\circled}[1]{\gencircled{#1}{$\bigcirc$}}
\newcommand{\cplus}{\,\gencircled{$+$}{$\bigcirc$}\,}
\newcommand{\cminus}{\,\gencircled{$-$}{$\bigcirc$}\,}

\newcommand{\vboxed}[1]{\gencircled{#1}{$\bigbox$}}

%\newcommand{\uneval}{\boxed{\makebox{1.0em}{}}}
\newcommand{\uneval}{\vboxed{~}}
\newcommand{\unevale}{\vboxed{\textcolor[rgb]{0.6,0.6,0.6}{\it e}}}
\newcommand{\unevalr}{\vboxed{\textcolor[rgb]{0.6,0.6,0.6}{\it r}}}
\newcommand{\unevall}{\vboxed{\textcolor[rgb]{0.6,0.6,0.6}{\it l}}}
%\newcommand{\node}{\raisebox{-0.5ex}{\rule{1.43ex}{1em}}}
\newcommand{\node}{\vboxed{{\footnotesize ?}}}
\newcommand{\aovId}{\Rfun{id}}
\newcommand{\rName}{\Rfun{ruleName}}
\newcommand{\fName}{\Rfun{funcName}}
\newcommand{\idName}{\Rfun{name}}
\newcommand{\bkg}{\Rfun{bkg}}
\newcommand{\vul}{\Rfun{vul}}
%\newcommand{\vulpos}{\vul(\pos)}
%\newcommand{\setvul}[3]{\vulpos:=(#1,#2,#3)}
%\newcommand{\setPos}{SetPosVUL}
%\newcommand{\SetPos}[1]{\Rroutine{\setPos}(#1)}
\newcommand{\sema}[1]{\ensuremath{[\![#1]\!]}}
\newcommand{\setvul}[3]{\sema{\pos}:=(#3, #2, #1)}

\newcommand{\Rfunction}{\mbox{\bf function }}
\newcommand{\Renum}{\mbox{\bf enum }}
\newcommand{\Runiverse}{\mbox{\bf universe }}


% ----------------- Partial Updates related Naming
\newcommand{\PartialUpdate}[1]{Special update#1}
\newcommand{\PartialUUpdate}[1]{Special Update#1}
\newcommand{\partialUpdate}[1]{special update#1}
\newcommand{\PARTIALUPDATE}[1]{SpecialUpdate#1}
\newcommand{\integra}[1]{aggrega#1}
\newcommand{\Integra}[1]{Aggrega#1}
\newcommand{\failedIntegra}[0]{failed\Integra{tion}}
\newcommand{\registry}[0]{updateInstructions}
\newcommand{\Registry}[0]{Registry}
\newcommand{\scratchpad}[0]{considered}
\newcommand{\uilva}[3]{\ensuremath{\langle #1, #2, #3 \rangle}}
\newcommand{\uima}[1]{\ensuremath{\langle\!\langle#1\rangle\!\rangle}}


% ----------------- Style and Formatting
\newcommand{\HIDE}[1]{}
\newcommand{\loc}[1]{\ensuremath{{}^{#1}}}
\newcommand{\lpatrule}[2]{\ensuremath{\llparenthesis\,{#1}\,\rrparenthesis}\tprsep\ensuremath{\rightarrow}\tprsep\ensuremath{#2}}
\newcommand{\tprsep}{\,}
\newcommand{\patrule}[2]{\lpatrule{#1}{\mbox{$\begin{array}[t]{l}#2\end{array}$}}}
\newcommand{\patrulebr}[3]{\patrule{#2}{}\\ \begin{minipage}{10cm}\hspace{#1}\bmRule #3 \emRule\end{minipage}}
\newenvironment{tabbedpatrule}%
    {\renewcommand{\tprsep}{&}\begin{tabular}{lcl}}%
    {\end{tabular}\renewcommand{\tprsep}{\,}}
\newcommand{\lpatrulepr}[3]{\ensuremath{\llparenthesis\,{#1}\,\rrparenthesis\raisebox{-2mm}{{\footnotesize [#2]}}}\tprsep\ensuremath{\rightarrow}\tprsep\ensuremath{#3}}
\newcommand{\patrulepr}[3]{\lpatrulepr{#1}{#2}{\mbox{$\begin{array}[t]{l}#3\end{array}$}}}
\newcommand{\patspace}{\\ \\}
\newcommand{\smallpatspace}{\vspace{0.15cm}\\}
\newcommand{\issue}[1]{[{\small Issue #1}]}
\newcommand{\ppoint}[1]{{\bf \footnotesize Plug-in Point:} #1}
\newcommand{\bref}{\begin{small} \vspace{0.2cm} \begin{tabular}{|l} \begin{minipage}{0.8\textwidth} \vspace{0.2cm} {\sf \textbf{Reference}}: \begin{list}{-}}
%
\newcommand{\eref}{\end{list} \vspace{0.2cm} \end{minipage} \end{tabular} \vspace{0.2cm} \end{small}}
\newcommand{\state}[1]{{\itshape #1}}

\newcommand{\equivdef}[2]{\mbox{$\begin{array}[t]{lcl}\mbox{$\begin{array}[t]{l}#1\end{array}$}&\ \equiv \ &\mbox{$\begin{array}[t]{l}#2\end{array}$}\end{array}$}}
\newcommand{\equivdefspace}{\\ \\}

\newcommand{\pluginRequires}{\vspace{2mm}\noindent{\bf Required: }}
\newcommand{\pluginFuncs}{\vspace{2mm}\noindent{\bf Functions: }}

% ----------------- Plugin Documentation Styles
\newenvironment{PluginHeader}[3]
%{\begin{boxitpara}{box 0.9 1 0.9 setrgbcolor fill}\begin{center}\begin{minipage}{0.92\textwidth}
{\begin{boxitpara}{box}\begin{center}\begin{minipage}{0.92\textwidth}
{\bf #1 Plug-in} \hfill {\small{version #2}} \\ 
\texttt{#3} \\
\rule[0.7ex]{\textwidth}{0.2mm}}
{\end{minipage}\end{center}\end{boxitpara}}
\newcommand{\plugininfo}[2]{\vspace{-0.8ex}~\\{\small{\bf #1:}} \\ \noindent\hspace*{0.5cm} {\em #2}}
\newcommand{\preqs}[1]{\plugininfo{Required Plugins}{#1}}
\newcommand{\pbkgs}[1]{\plugininfo{Backgrounds}{#1}}
\newcommand{\punivs}[1]{\plugininfo{Universes}{#1}}
\newcommand{\pfuncs}[1]{\plugininfo{Functions}{#1}}
\newcommand{\poprs}[1]{\plugininfo{Operators}{{\texttt{#1}}}}
\newcommand{\pacts}[1]{\plugininfo{Update Actions}{#1}}
\newcommand{\psrcmodes}[1]{\plugininfo{Source Modes}{#1}}
\newcommand{\ptrgmodes}[1]{\plugininfo{Target Modes}{#1}}
\newcommand{\pplugins}[1]{\plugininfo{Enclosed Plugins}{#1}}

% ----------------- Symbols
\newcommand{\mset}[1]{\{\hspace{-0.27em}| #1 |\hspace{-0.28em}\}}
\newcommand{\collection}[1]{[\hspace{-0.27em}\ #1\ \hspace{-0.28em}]}

% ---------------------------------------------------------
% CoreASMCommands.tex  (last modified: $Date: 2009-07-30 15:18:35 +0200 (Thu, 30 Jul 2009) $)
%
% A series of LaTeX commands defined for 
% typesetting CoreASM specifications
%
% Copyright (c) 2009 Roozbeh Farahbod
%
% This work is licensed under the Creative Commons 
% Attribution-NonCommercial-NoDerivs License. To view 
% a copy of this license, visit the following link:
%
%   http://creativecommons.org/licenses/by-nc-nd/2.0/ca/ 
% 
% ---------------------------------------------------------

\newcommand{\clkeyword}[1]{\textcolor[rgb]{0.5,0,0.3}{{\bfseries #1}}}
\newcommand{\clfn}[1]{\textcolor[rgb]{0,0,0.5}{#1}}
\newcommand{\clcomment}[1]{\textcolor[rgb]{0.3,0.3,0.3}{#1}}
\newcommand{\clstr}[1]{\textcolor[rgb]{0,0.4,0}{#1}}
\newcommand{\cldblsp}{\hspace*{1em}}
\newcommand{\cltb}{\cldblsp\cldblsp}

% COREASM Environment
\newenvironment{coreasm}{\begin{quote}
\renewcommand{\baselinestretch}{0.8}
\ttfamily
\small
}{\normalsize
\renewcommand{\baselinestretch}{1}
\normalfont
\end{quote}
}

% 


% ----------------- CoreASM Spec Float
\floatstyle{boxed}
\newfloat{coreasmbox}{thp}{los}[section]
\floatname{coreasmbox}{CoreASM Spec}
\floatstyle{plain}

\makeindex

\pagestyle{fancy}
%\renewcommand{\chaptermark}[1]{
%	\markboth{\thechapter.\ #1}{}}
\lhead{{\small \leftmark}}
\rhead{{\small \CoreASM\ {\scshape Language User Manual}}}
% You need to turn the 'twoside' option on, 
% before uncommenting the following lines.
%\fancyhead{}
%\fancyhead[LO,RE]{{\small \leftmark}}
%\fancyhead[RO,LE]{{\small \CoreASM Execution Engine}}



% ----------------- Commands
\newcommand{\hlight}[1]{\textcolor{blue}{\textbf{#1}}}
\newcommand{\edit}[1]{\textcolor[rgb]{0, 0.5, 0}{#1}}
\newcommand{\new}[1]{1}%\edit{#1}}
\newcommand{\comment}[1]{\textcolor[rgb]{0, 0, 0.5}{Comment: #1}}
\newcommand{\incorrect}[1]{\textcolor[rgb]{0.5, 0, 0}{#1}}

\newcommand{\codebf}[1]{\xspace{\bf \code{#1}}\xspace}
\newcommand{\keyword}[1]{\codebf{#1}}

\newcommand{\codecom}[1]{\textcolor[gray]{0.5}{// #1}}
%\newcommand{\ruleform}[1]{\vspace*{4mm} \noindent \framebox{\sffamily \footnotesize \bfseries R} #1 \vspace{1mm}}
\newcommand{\ruleform}[2]{\pform{$\blacktriangleright$}{#1}{#2}}
\newcommand{\funcform}[2]{\pform{$\blacklozenge$}{#1}{#2}}
\newcommand{\opform}[2]{\pform{$\vartriangleright$}{#1}{#2}}
\newcommand{\pform}[3]{\vspace*{4mm} \noindent #1 #2 \vspace{1mm}\textcolor[gray]{0.7}{\dotfill}\mbox{{\sffamily \footnotesize #3}}}
%\newcommand{\bExample}{\begin{quote}\ttfamily}
%\newcommand{\eExample}{\end{quote}}
\newcommand{\bExample}{\begin{small} \vspace{0.3cm} \begin{tabular}{|l} \begin{minipage}{0.85\textwidth} \vspace{0.2cm} \ttfamily}
\newcommand{\eExample}{\vspace{0.2cm} \end{minipage} \end{tabular} \vspace{0.3cm} \end{small}}
\newenvironment{spec}{\ttfamily}{}
%\newcommand{\optional}[1]{\raisebox{-4mm}{$\stackrel{\underline{\mbox{#1}}}{\mbox{\footnotesize optional}}$}}
\newcommand{\optional}[1]{$\underset{optional}{\underline{\mbox{#1}}}$}
%\newcommand{\optional}[1]{$\underbrace{\mbox{#1}}_{optional}$}
%\newcommand{\optional}[1]{$[\mbox{\code{#1}}]$}
\renewcommand{\id}{{\em id}\xspace}
\renewcommand{\loc}{{\em loc}\xspace}
\newcommand{\val}{{\em value}\xspace}
\newcommand{\crule}{{\em rule}\xspace}
\newcommand{\guard}{{\em guard}\xspace}
\newcommand{\rulei}[1]{{\em rule}$_{#1}$\xspace}
\newcommand{\idi}[1]{{\em id}$_{#1}$\xspace}
\newcommand{\valuei}[1]{{\em value}$_{#1}$\xspace}
\newcommand{\valuepi}[1]{{\em value}$'_{#1}$\xspace}
\newcommand{\loci}[1]{{\em loc}$_{#1}$\xspace}
%\newcommand{\tab}{$~~~~$}
\newcommand{\tab}{\textcolor[gray]{1}{MMM}}
%\newcommand{\tab}{\code{\hspace*{1em}}}
\newcommand{\specref}[1]{Spec~\ref{spec:#1}}
\newcommand{\indexrule}[1]{\index{#1 rule@\keyword{#1} rule}}
\newcommand{\indexexp}[1]{\index{#1 expression@\keyword{#1} expression}}
\newcommand{\indexfunc}[1]{\index{#1 function@\code{#1} function}}

% ----------------- Commands: Copyright Notice 
\newcommand{\copyrightNotice}[1]{{Copyright \copyright\ #1}}
%\newcommand{\revision}{{\footnotesize Document \mbox{$Revision: #1 $}}}


% ----------------- Title and Abstract

\renewcommand{\url}[1]{\href{http://#1}{{\ttfamily #1}}}

\usepackage{hyperref}
\begin{document}

\title{\huge \CoreASM Language User Manual \\ {\Large engine version \version} \\ {\large \url{github.com/CoreASM/}}}

\author{Roozbeh Farahbod \\ \texttt{info@coreasm.org}\\\\Marcel Dausend \\ \texttt{marcel.dausend@uni-ulm.de}}  

\date{\copyrightNotice{2006-2014} \\~\\ {\bf DRAFT of \today} \\ {This document is still under construction to match the latest version of the engine. Your criticism is welcome!}}

 \maketitle

\newpage
\tableofcontents
\newpage

% ----------------- Main Body
\PScommands

\section{Installing \CoreASM}

There are currently two user interfaces available for the \CoreASM engine: a
command-line interface called \Carma, and a graphical interactive
development environment in the Eclipse platform, known as the \CoreASM Eclipse
Plugin.

% Since the specifics of how to install plugins on Eclipse can vary from version to version, 
% please see -website- for up-to-date information" ecc.
% For instructions on how to download and install and use \CoreASM with any of these
% interfaces, please visit \url{www.coreasm.org/download}.

\subsection{\CoreASM with \Carma}

\CoreASM engine with \Carma can be downloaded from the web at \url{www.coreasm.org/download}.


\subsubsection{System Requirements}

You need to have Sun Microsystems Java 1.6 (JVM) installed on your machine.

\subsubsection{Installing \Carma}

To install \CoreASM with \Carma just unzip the contents of the binary package into a 
directory of your choice. Alternatively, You can build \CoreASM with \Carma using the 
ant build file provided in the source package.

\subsubsection{Running \Carma}

Under \Carma's home directory (where you installed Carma), 
simply run `carma' (under POSIX systems) or `carma.bat' (under Windows systems). 
To be able to run \Carma form other directories, change the value of {\ttfamily CARMA\_HOME} environment 
variable in `carma' or `carma.bat' (depending on your operating system) so that it points to the folder 
in which Carma is installed.

To start, try Carma with `{\ttfamily --help}' to see the list of command-line arguments. 

\subsection{\CoreASM Eclipse Plugin}

This section explains how to install the \CoreASM Eclipse plugin.

\subsubsection{System Requirements}

You need to have Eclipse 3.5 or newer installed on your machine.

\subsubsection{Installing \CoreASM Eclipse Plugin}

To install the latest development version of the \CoreASM Eclipse plugin using the update site, 
run Eclipse and follow these steps:

\newcommand{\menulink}{$\rightarrow$}
\newcommand{\menunav}[1]{{\sl #1}}

\begin{enumerate}
    \item Uninstall and remove any previously installed \CoreASM Eclipse plugin version 0.4.x or older.
    
	\item Under \menunav{Help \menulink Install New Software\ldots} add an update site with the following information:
	\begin{quote}
        {\em Name:} CoreASM Eclipse Plugin trunk

        {\em URL:} \url{http://www.coreasm.org/eclipse/update-trunk}
	\end{quote}
	
	\item Select the ``CoreASM Eclipse Plugin Features'' category and continue with the installation. 
\end{enumerate}

The development version of the \CoreASM Eclipse plugin is tested on Eclipse 3.5.

\subsubsection{Using \CoreASM Eclipse Plugin}

\paragraph{Creating a New Project} ~

\begin{enumerate}
    \item From the Eclipse menu choose: \menunav{File \menulink New \menulink Project\ldots}
    \item Choose \menunav{General \menulink Project} from the ``New Project'' dialog. Click \menunav{Next}.
    \item Give the project a name. Click \menunav{Finish}. 
\end{enumerate}

\paragraph{Creating a New \CoreASM Specification} ~

\noindent {\em Method 1:}

\begin{enumerate}
	\item From the Eclipse menu choose: \menunav{File \menulink New \menulink Other\ldots}
	\item In the New dialog choose \menunav{CoreASM \menulink CoreASM Specification}. Click \menunav{Next}.
    \item Choose the project container for the specification.
    \item Enter the name of the new \CoreASM specification file. The file must have the extension {\ttfamily .casm} or {\ttfamily .coreasm}. 
	\item Click \menunav{Finish}. 
\end{enumerate}

\noindent {\em Method 2:}

\begin{enumerate}
    \item File the Eclipse menu choose: \menunav{File \menulink New \menulink File\ldots}
    \item In the new file dialog choose a project container for the new file and enter a name for the new file. Again, The file must have the extension {\ttfamily .casm} or {\ttfamily .coreasm}.
    \item Click \menunav{Finish}. 
\end{enumerate}

\paragraph{Running a \CoreASM Specification} ~

\noindent {\em Method 1:}

Shortcut method for running a specification with default configuration:

\begin{enumerate}
    \item In the Eclipse window, right click on a \CoreASM specification file.
    \item In the context menu choose: \menunav{Run as \menulink CoreASM Specification} 
\end{enumerate}

\noindent {\em Method 2:}

If you need more control of the parameters for repeated execution, 
you can create a specific \CoreASM Launch Configuration as follows:

\begin{enumerate}
    \item From the Eclipse menu choose: \menunav{Run \menulink Run\ldots}
	\item In the ``Run'' dialog, choose the ``ASM Specification'' launch configuration
group and create a new ASM launch configuration (right click then select \menunav{New}, or
click the New launch configuration button on the tool bar).
    \item Enter a name for the launch configuration.
    \item Enter the project and specification file to be run. This can be done via the browse buttons.
    \item Configure the ``Termination Conditions'' and ``Output Verbosity'' options as desired.
    \item Click \menunav{Apply}.
    \item Click \menunav{Run} to run the specification. 
\end{enumerate}

Once the configuration has been launched once, it can be run again through the Run Button/Drop down menu in the main Eclipse toolbar.

\paragraph{Controlling the Execution of the \CoreASM Engine} ~

While the engine is running, you can click on the 
``Stop CoreASM Engine'' button to stop the run. To pause a running engine, 
click on the ``Pause CoreASM Engine'' button. If you pause the engine, the 
run can be resumed by clicking on the ``Resume CoreASM Engine'' button. 

% ----------------- Section: CoreASM Specification
\section{\CoreASM Specification}


Figure~\ref{fig:specStructure} shows a typical structure of a \CoreASM specification\footnote{
As of version 1.1, this structure is not required anymore and different components of the specification
can appear in any order. The only requirement is that the specification must start with a \keyword{CoreASM}
phrase.}.
Every specification starts with the keyword \keyword{CoreASM}\index{CoreASM@\keyword{CoreASM}} followed by the name
of the specification. Plugins that are required in the specification are then listed one by 
one with the keyword \keyword{use}\index{use@\keyword{use}} followed by the name of the plugin. 

The \emph{Header}\index{Header block} block is where various definitions take place. What goes into this  
section depends on the plugins that are used. The \CoreASM Kernel 
does not define anything for the header section. 

The \emph{init rule}\index{init rule} of the specification (the rule
that creates the initial state) is defined by keyword 
\keyword{init}\index{init@\keyword{init}|see{init rule}} followed
by a rule name. This would be the rule that initializes the state of the machine
that is defined by the specification. The body of the init rule must be declared
in the \emph{Rule Declaration}\index{Rule Declaration} block.

A sample \CoreASM specification is presented in \specref{thisiscoreasm}.

\begin{figure}
\begin{center}
\begin{minipage}{0.55\textwidth}
\begin{boxitpara}{box}
\small
\codebf{CoreASM} \code{SpecificationName}\\

\codebf{use} \code{SamplePlugin}\\
\codebf{use} \ldots
\begin{center} \begin{minipage}{0.97\textwidth} \begin{boxitpara}{box 0.9 setgray fill}
\sf Header Block
\\
\end{boxitpara} \end{minipage} \end{center}
\codebf{init} \code{InitRuleName}
\begin{center} \begin{minipage}{0.97\textwidth} \begin{boxitpara}{box 0.9 setgray fill}
\sf Rule Declaration Block
\\\vspace{1.5cm}
\end{boxitpara} \end{minipage} \end{center}
\end{boxitpara}
\end{minipage}
\end{center}
\caption{Typical Structure of a \CoreASM Specification}
\label{fig:specStructure}
\end{figure}

\subsection{Running \CoreASM Specifications}
\label{running}

To run a \CoreASM specification you need to have a \CoreASM engine driver. Currently, there are
two engine drivers available:

\begin{itemize}
	\item \CoreASM Eclipse Plugin is a plugin for the Eclipse (see \url{www.eclipse.org}) 
		development environment that provides syntax highlighting and a nice GUI to control 
		specification runs.

	\item \Carma is a command-line \CoreASM engine driver. To run a specification using 
		\Carma simply run \Carma on the command line and pass it the name of the specification 
		file as an argument. Make sure to specify a termination condition (e.g., \code{--steps 20}
		or \code{--empty-updates}) for the run.
		Run \Carma with \code{--help} for a complete list of options that controls its behavior.  

		The following command runs \code{MySpec} using \Carma and stops after 30 steps, or after
		a step that generates empty updates; it also dumps the final state before termination.

		\begin{quote}\code{carma --steps 30 --empty-updates --dump-final-state MySpec.coreasm} \end{quote} 

		Alternatively, to run the specification of \specref{thisiscoreasm}, one can use the following options which would make \Carma to mark the end of each step and stop after 30 steps or when there is no agent with a defined program:

		\begin{quote}\code{carma --marksteps --steps 30 --no-agent ThisIsCoreASM.coreasm} \end{quote} 

		In this example, \Carma will stop after three steps.

		%Here are some examples to run 
		%\Carma with a sample specification:
\end{itemize}
%\subsection{This Is CoreASM}
%\label{thisiscoreasm}

\begin{coreasmbox}
	\begin{spec}
	\codebf{CoreASM} ThisIsCoreASM\\
	\\
	\codebf{use} Standard\\
	\\
	\codebf{init} InitRule\\
	\\
	\codebf{rule} InitRule = \\
	\tab \codebf{par}\\
	\tab \tab terminate := \codebf{false}\\
	\tab \tab program(self) := @MainProgram\\
	\tab \codebf{endpar}\\
	\\
	\codebf{rule} MainProgram = \\
	\tab \codebf{if} \codebf{not} terminate \codebf{then}\\
	\tab \tab \codebf{par}\\
	\tab \tab \tab \codebf{print} "This is CoreASM."\\
	\tab \tab \tab terminate := \codebf{true}\\
	\tab \tab \codebf{endpar}\\
	\tab \codebf{else}\\
	\tab \tab program(self) := \codebf{undef}
	\end{spec}
	\caption{CoreASM-Says-Hello example}
	\label{spec:thisiscoreasm}
\end{coreasmbox}


% -------------------------------- * KERNEL
\section{Kernel}
\label{kernel}

Kernel\index{CoreASM kernel}\index{kernel|see{CoreASM kernel}} of the \CoreASM engine provides the minimum set of vocabulary and rules to 
have a \CoreASM specification.

Basic values such as \keyword{undef}\index{undef@\keyword{undef}}, \keyword{true}\index{true@\keyword{true}}, and \keyword{false}\index{false@\keyword{false}} are defined in the 
kernel along with the background of Boolean values (\code{BOOLEAN})\index{Boolean background} and the universe 
of \code{Agents}\index{Agents@\code{Agents}}. A function called \code{program}\indexfunc{program} is also defined in the kernel which 
maps agents to their programs (\CoreASM rules). At any time during the evaluation of a 
rule, \keyword{self}\index{self@\keyword{self}} refers the the agent that is running the enclosing rule.

\CoreASM kernel also defines a couple of important operators:

\opform{\valuei{1} = \valuei{2}}{Kernel}

This is the equality operator\index{equality operator}\index{=@=|see{equality operator}}. 

\opform{\codebf{ruleelement} \id}{Kernel}\index{ruleelement@\keyword{ruleelement}}

This operator returns the rule element of a rule with the given name (\id). \new{Rule element is an element in the CoreASM state
that represents a rule defined in the specification.} It is useful in 
assigning rules to programs of agents. In the following example, \code{Main} is the name 
of a rule:

\bExample
	program(self) := \codebf{ruleelement} Main
\eExample

The above rule, assigns the rule named \code{Main} as the value of the program 
of the agent running this rule.

\opform{{\bf @} \id}{Kernel}\index{atsign@\code{@} sign}

Returns the rule element (rule body) or function element of a rule or function 
with the given name (\id). If the given name is the name of a rule, it works exactly 
the same as \keyword{ruleelement}. Thus, if \code{Main} is rule, we can have:

\bExample
	program(self) := {\bf @}Main
\eExample

\subsection{Rule Forms}

The following rule forms are defined in the kernel:

\ruleform{\loc \code{:=} \val}{Kernel}\index{update rule}\index{:=@\code{:=}|see{update rule}}

Assigns the value of \val to the location \loc.

\ruleform{\codebf{import} \id\ \codebf{do}\xspace \crule}{Kernel}\index{import@\keyword{import}}

Imports a new element, assigns it as the value of the environment variable \id,
and evaluates \crule.

\ruleform{\codebf{skip}}{Kernel}\indexrule{skip}

Does nothing. This is like a NoOp.
  

\subsection{Kernel Engine Properties}

The following properties affect the behaivour of the \CoreASM engine. 

\begin{description}
	\item[engine.error.printStackTrace] if equals to \code{"yes"}, the engine will 
		print the stack trace of errors and exceptions. The default value is \code{"no"}. 

	\item[engine.limits.maxProcessors] the maximum number of processors the engine 
		can use for simulation. The default value is \code{"1"}. 

	\item[scheduler.printProcessorStats] if equals to \code{"yes"}, the engine 
		will print some information on processor utilization after every step. The default value is \code{"no"}. 

	\item[scheduler.threadBatchSize] in a multi-threaded simulation, the value
		of this property defines the minimum number of agents assigned to every thread.
	The default value is \code{"1"}.
	
	\item[engine.pluginFolders] a colon-separated list of folders that provide additional plugins. 

	\item[engine.pluginLoadRequest] a comma separated list of plugins to be loaded in addition to those listed in the specification being loaded.
\end{description}

% -------------------------------- * BASIC ASM PLUGINS
\section{Basic ASM Plugins}\index{Basic ASM plugin}
\label{basicASM}

In this section we list the plugins that provide the basic ASM rule forms. All
the plugins in this section can be loaded individually (as instructed in each
section) or all together with the following \keyword{use} phrase, 

\bExample
	\codebf{use} BasicASMPlugins
\eExample

\noindent which automatically loads the following plugins: BlockRule,
ConditionalRule, ChooseRule, ForallRule, LetRule, and
Number.

Note that the words ``Plugin'' and ``Plugins'' in the name of the plugins are optional.
For example, Basic ASM plugins can also be loaded using the following line:

\bExample
	\codebf{use} BasicASM
\eExample


% -------------------------------- BLOCK RULE
\subsection{Block Rule}
\label{block}

The Block Rule plugin\index{Block Rule plugin} can be loaded by the following \keyword{use} phrase:

\bExample
	\codebf{use} BlockRule
\eExample

\noindent This plugin provides the following rule form:

\ruleform{\codebf{par} \rulei{1} \optional{\rulei{2} \ldots \rulei{n}} \codebf{endpar}}{Block Rule Plugin}\index{block rule}\index{par@\keyword{par}|see{block rule}}

Instructs the engine to evaluate all the given rules in parallel. The update 
generated by this rule is the union of all the updates generated by \rulei{1}
to \rulei{n}. %A block rule must have at least two rules ($n \geq 2$).


% -------------------------------- CHOOSE
\subsection{Choose Rule}
\label{choose}

The Choose Rule plugin\index{Choose Rule plugin} can be loaded by the following \keyword{use} phrase:

\bExample
	\codebf{use} ChooseRule
\eExample

\noindent This plugin provides the following rule form:

\ruleform{\codebf{choose} \id \codebf{in} \val \optional{\codebf{with} \guard} \codebf{do} \rulei{1} 
		\optional{\codebf{ifnone} \rulei{2}} \optional{$~$\codebf{endchoose}${~}$}}{Choose Rule Plugin}\index{choose rule@\keyword{choose} rule}

Chooses an element from the {\em enumerable}\index{enumerable}\footnote{An {\em enumerable} is an element
that can be enumerated; i.e., that is a collection of other values. Sets, universes, and 
some backgrounds are enumerable values.} 
\val which
satisfies \guard, assigns it as the value of \id and evaluates \rulei{1}.  If
the \keyword{ifnone} clause is provided, \rulei{2} will be evaluated if no
element can be found. The last keyword \keyword{endchoose} is optional. 

The following example chooses the minimum price $p$ from the set of $prices$ and prints (see Section~\ref{io}) 
$p$ on the screen:

\bExample
	\codebf{choose} p \codebf{in} prices \codebf{with} 
		(\codebf{forall} pi \codebf{in} prices \codebf{holds} p <= pi) \codebf{do}\indexrule{choose} \\
	\tab \codebf{print} p
\eExample

See Section~\ref{predicate} for more information on \keyword{forall} expressions.

The ChooseRule plugin also provides the following expression form to non-deterministically {\em pick} a value from
an enumerable that satisfies the given (optional) condition:

\opform{\codebf{pick} \id \codebf{in} \val \optional{\codebf{with} \guard}}{Choose Rule Plugin}\indexexp{pick}

\noindent For example, the following assignment non-deterministically assigns true or false to location \code{foo}:

\bExample
	foo := \codebf{pick} x \codebf{in} \{\codebf{true}, \codebf{false}\}
\eExample

% -------------------------------- CONDITIONAL
\subsection{Conditional Rule}
\label{conditional}

The Conditional Rule plugin\index{Conditional Rule plugin} can be loaded by the following \keyword{use} phrase:

\bExample
	\codebf{use} ConditionalRule
\eExample

\noindent This plugin provides the following rule forms:

\ruleform{\codebf{if} \val \codebf{then} \crule}{Conditional Rule Plugin}\index{conditional rule}\index{if@\keyword{if-then-else} rule|see{conditional rule}}

Evaluates \crule only if \val is \code{true}. It expects \val to be
a Boolean value (being either \code{true} or \code{false}).

\ruleform{\codebf{if} \val \codebf{then} \rulei{1} \codebf{else} \rulei{2}}{Conditional Rule Plugin}\index{conditional rule}

Evaluates \rulei{1} only if \val is \code{true} and \rulei{2} only if \val
is \code{false}. It expects \val to be
a Boolean value (being either \code{true} or \code{false}).

The Conditional Rule plugin also provides a conditional operation of the form:

\opform{\valuei{c} \codebf{?} \valuei{t} \codebf{:} \valuei{f}}{Conditional Rule Plugin}\index{conditional operation}

The value of this operator is \valuei{t}, if \valuei{c} evaluates to \code{true}; it is \valuei{f},
if \valuei{c} evaluates to \code{false}; otherwise, it is \keyword{undef}. 


% -------------------------------- FORALL
\subsection{Forall Rule}
\label{forall}

The Forall Rule plugin\index{Forall Rule plugin} can be loaded by the following \keyword{use} phrase:

\bExample
	\codebf{use} ForallRule
\eExample

\noindent This plugin provides the following rule form:

\ruleform{\codebf{forall} \id \codebf{in} \val \optional{\codebf{with} \guard} 
		\codebf{do} \crule \optional{$~$\codebf{endforall}$_{~}$}}{Forall Rule Plugin}\index{forall rule@\keyword{forall} rule}

For all the elements in the enumerable \val that satisfy \guard, assigns the
element to \id, and evaluates \crule.
The following examples assigns the \code{DefaultProgram} rule as the program of
all the agents program of which is \keyword{undef}:

\bExample
	\codebf{forall} a \codebf{in} Agents \codebf{with} program(a) = \codebf{undef} \codebf{do}\\
	\tab program(a) := \codebf{ruleelement} DefaultProgram
\eExample


% -------------------------------- LET
\subsection{Let Rule}
\label{let}

The Let Rule plugin\index{Let Rule plugin} can be loaded by the following \keyword{use} phrase:

\bExample
	\codebf{use} LetRule
\eExample

\noindent This plugin provides the following rule form:

\ruleform{\codebf{let} \idi{1} = \valuei{1} \optional{, \idi{2} = \valuei{2} , 
		\ldots , \idi{n} = \valuei{n}} \codebf{in} \crule }{Let Rule Plugin}\index{let rule@\keyword{let} rule}

For all the given pairs of \id and \val, assigns \valuei{i} as the value of
the environment variable \idi{i}, and evaluates \crule.
 

% -------------------------------- CASE
\subsection{Case Rule}
\label{case}

The Case Rule plugin\index{Case Rule plugin} can be loaded by the following \keyword{use} phrase:

\bExample
	\codebf{use} CaseRule
\eExample

\noindent This plugin provides the following rule form:

\ruleform{\codebf{case} \val \codebf{of} \valuei{1} \code{:} \rulei{1}\ldots \valuei{n} \code{:} \rulei{n} \codebf{endcase}}{Case Rule Plugin}\index{case rule@\keyword{case} rule}

The case condition \val will be evaluated first and then
all the guards \valuei{i} will be evaluated in an unspecified order. Afterward, rules
with a guard value equal to the value of the case condition will be evaluated.
Finally, the updates generated by the matching cases are united to form the
set of updates generated by the case rule.
 
% -------------------------------- PREDICATE LOGIC
\subsection{Predicate Logic}
\label{predicate}

The Predicate Logic plugin\index{Predicate Logic plugin} can be loaded by the following \keyword{use} phrase:

\bExample
	\codebf{use} PredicateLogic
\eExample

\noindent This plugin provides the following functions and expression forms: 

\opform{\codebf{forall} \id \codebf{in} \val \codebf{holds} \guard}{Predicate Logic Plugin}\indexexp{forall}

This Boolean expression holds if \guard holds for all the elements of \val (which must be an enumerable value). 

\opform{\codebf{exists} \id \codebf{in} \val \codebf{with} \guard}{Predicate Logic Plugin}\indexexp{exists}

This Boolean expression holds if there exists at least one element in \val
(which must be an enumerable value) that satisfies \guard.

\opform{\valuei{1} \code{!=} \valuei{2}}{Predicate Logic Plugin}\index{not-equal operator}\index{not equal@\code{!=} operator|see{not-equal operator}}

This is the not-equal operator which is defined on all elements. The semantics of 
this operator is equivalent to `\code{\codebf{not} (value1 = value2)}'. 

\opform{\valuei{1} bin-op \valuei{2}}{Predicate Logic Plugin}\index{Boolean operators}

Performs a binary operation on the given values. The following operators are defined
on Boolean values:

\begin{quote} \code{or}, \code{xor}, \code{and}, \code{implies} \end{quote}\index{Boolean operators}\index{or@\keyword{or}|see{Boolean operators}}\index{xor@\keyword{xor}|see{Boolean operators}}\index{and@\keyword{and}|see{Boolean operators}}\index{implies@\keyword{implies}|see{Boolean operators}}

\noindent The following two operators are also defined which require \valuei{2} to be an enumerable:

\begin{quote} \code{memberof} and \code{notmemberof} \end{quote}\index{membership operators}\index{memberof@\keyword{memberof} operator|see{membership operators}} 

\opform{\codebf{not} \val}{Predicate Logic Plugin}\index{not@\keyword{not} operator}

This is the negation operator which is defined on Boolean values.



% -------------------------------- NUMBER
\subsection{Number Background}
\label{number}

The Number plugin\index{Number plugin} can be loaded by the following \keyword{use} phrase:

\bExample
	\codebf{use} Number
\eExample

\noindent This plugin provides the number background (\code{NUMBER})\index{Number background} and
a valuable set of functions and expression forms. 

\opform{\valuei{1} bin-op \valuei{2} }{Number Plugin}

Performs binary operations on number values. Currently supported operators are

\begin{quote} \code{+ - * / div \%} \\ \code{ > >= < <= } \end{quote}

which result in Number or Boolean values.

\opform{$|$ \val $|$ }{Number Plugin}\index{size-of operator}\index{=bar@\texttt{"|x"|}|see{size-of operator}}

If \val is enumerable (such as a set), this operator will evaluate to the size of \val.

\funcform{$\Rfun{infinity} : \rightarrow \Rdom{Number}$}{Number Plugin}\index{infinity@\code{infinity} function}

Is the positive infinity.

\funcform{$\Rfun{toNumber} : \Rdom{Element} \rightarrow \Rdom{Number}$}{Number Plugin}\index{toNumber@\code{toNumber} function}

This is a conversion function that maps any value to a Number value (which can also be \keyword{undef}).
The following example uses this function to read a number from the environment:

\bExample
	\codebf{seq} \\
	\tab amount := input("Input Amount")\\
	\codebf{next} \\
	\tab \codebf{let} value = toNumber(amount) \codebf{in} \\
	\tab \tab \codebf{if} value = \codebf{undef} \codebf{then} \\
	\tab \tab \tab \codebf{print} "Error."\\
	\tab \tab \codebf{else} \\
	\tab \tab \tab DepositAmount(value)\\
\eExample

\funcform{$\Rfun{isNaturalNumber} : \Rdom{Number} \rightarrow \Rdom{Boolean}$}{Number Plugin}\index{isNaturalNumber@\code{isNaturalNumber} function}

Returns \code{true} if the argument is a Natural number (i.e., positive non-zero integer).

\funcform{$\Rfun{isIntegerNumber} : \Rdom{Number} \rightarrow \Rdom{Boolean}$}{Number Plugin}\index{isIntegerNumber@\code{isIntegerNumber} function}

Returns \code{true} if the argument is an Integer number.

\funcform{$\Rfun{isRealNumber} : \Rdom{Number} \rightarrow \Rdom{Boolean}$}{Number Plugin}\index{isRealNumber@\code{isRealNumber} function}

Returns \code{true} if the argument is a valid non-infinite Real number.

\funcform{$\Rfun{isEvenNumber} : \Rdom{Number} \rightarrow \Rdom{Boolean}$}{Number Plugin}\index{isEvenNumber@\code{isEvenNumber} function}

Returns \code{true} if the argument is an Integer number divisible by two.

\funcform{$\Rfun{isOddNumber} : \Rdom{Number} \rightarrow \Rdom{Boolean}$}{Number Plugin}\index{isOddNumber@\code{isOddNumber} function}

Returns \code{true} if the argument is an Integer number which is not divisible by two.

\funcform{$\mbox{\code{size}} : \Rdom{Element} \rightarrow \Rdom{Number}$}{Number Plugin}\index{size function@\code{size} function}

Returns the size of the given collection.

\smallskip
The Number plugin also provides a background for number ranges (\code{NUMBER\_RANGE})\index{Number Range background}.
Number range elements are enumerable and can be defined using the following syntax.

\opform{\code{[ \valuei{start} .. \valuei{end} \optional{ \codebf{step} \valuei{step}} ] }}{Number Plugin}\index{number range elements}\index{\{x\}@\texttt{"[ "]}|see{number range elements}}

Creates a range of numbers from \valuei{start} to \valuei{end} with the optional step. It is 
also possible to use `\code{:}' instead of \keyword{step}.
In the following example, \code{RandomGuess} returns a random number between 1 and 100:

\bExample
	\codebf{rule} RandomGuess = \\
	\tab \codebf{return} rand \codebf{in}\\
	\tab \tab \codebf{choose} x \codebf{in} [ 1 .. 100 ] \codebf{do}\\
	\tab \tab \tab rand := x
\eExample


% -------------------------------- * COREASM PLUGINS
\section{Standard Plugins}\index{Standard plugins}
\label{standard}

Most of the \CoreASM plugins, including all the Basic ASM plugins, are included in the Standard plugins package. 
In this section we list the plugins that are provided by the Standard plugins package in addition to the ones 
listed in the previous section. All
these plugins can be loaded individually (as instructed in each
section) or all together with the following \keyword{use} phrase, 

\bExample
	\codebf{use} Standard
\eExample

\noindent which automatically loads all the plugins listed in Section~\ref{basicASM} 
in addition to the ones listed in this section.


% -------------------------------- KERNEL EXTENSIONS
\subsection{Kernel Extensions}
\label{kernelext}

The Kernel Extensions plugin\index{Kernel Extensions plugin} can be loaded by the following \keyword{use} phrase:

\bExample
	\codebf{use} KernelExtensions
\eExample

This plugin extends the Kernel capabilities in handling function and rule elements. The current version provides the following expression and rule forms. 

\opform{\id (\valuei{1},\ldots, \valuei{n}) (\valuepi{1}, \ldots, \valuepi{m})}{Kernel Extensions Plugin}\indexexp{application}\vspace{-2mm}

\opform{(\val) (\valuepi{1}, \ldots, \valuepi{m})}{Kernel Extensions Plugin}\indexexp{application}

The above two forms apply the arguments \valuepi{i} to the function element at location \id(\valuei{1},\ldots, \valuei{n}) or to 
the function element resulting from evaluation of \val. If the function element refers to a function in the state, 
the location of the above expressions are also set to the location of the function with the given arguments; otherwise (e.g., in case 
of non-state functions) the location will be not be defined. Here are some examples, assuming that \code{foo} and \code{bar} are two 
defined functions, and \code{bar = @foo}:

\bExample
	\clkeyword{print} bar()(5, 4)   \clcomment{// printing the value of foo(5, 4)}\\
	(bar)(1, 3) := 4   \clcomment{// assigning 4 to foo(1, 3) }\\ 
\eExample

\ruleform{\codebf{call} \id (\valuei{1},\ldots, \valuei{n}) (\valuepi{1}, \ldots, \valuepi{m})}{Kernel Extensions Plugin}\indexexp{call} \vspace{-2mm}
 
\ruleform{\codebf{call} (\val) (\valuepi{1}, \ldots, \valuepi{m})}{Kernel Extensions Plugin}\indexexp{call}

The above two rules call the rule element value of \id(\valuei{1},\ldots, \valuei{n}) (the first form) or \val (the second form) 
with the arguments \valuepi{i}. For example, if we have \code{foo(5) = @MyRule} and 

\bExample
	\clkeyword{rule} MyRule(a, b) = \\
	\tab \clkeyword{print} a + \clstr{" talks to "} + b\\
\eExample

then we can call this rule by:

\bExample
	\clkeyword{call} foo(5) (\clstr{"John"}, \clstr{"Mary"}) \clcomment{// prints "John talks to Mary"}
\eExample

This plugin is not yet part of the Standard Plugin package.


% -------------------------------- ABSTRACT
\subsection{Abstraction}
\label{abstraction}

The Abstraction plugin\index{Abstraction plugin} can be loaded by the following \keyword{use} phrase:

\bExample
	\codebf{use} Abstraction
\eExample

This plugin provides the following rule form, which is useful when the specifier wants
to leave the detail of a rule abstract. 

\ruleform{\codebf{abstract} \val}{Abstraction Plugin}\indexrule{abstract}

In the following example, the rule \code{SendMessage} is left abstract:

\bExample
	\codebf{rule} SendMessage =\\
	\tab \codebf{abstract} "Sending the message."\\ 
\eExample


% -------------------------------- EXTEND
\subsection{Extend Rule}
\label{extend}

The Extend Rule plugin\index{Extend Rule plugin} can be loaded by the following \keyword{use} phrase:

\bExample
	\codebf{use} ExtendRule
\eExample

\noindent This plugin provides the following rule form:

\ruleform{\codebf{extend} \val \codebf{with} \id \codebf{do} \crule}{Extend Rule Plugin}\index{extend rule@\keyword{extend} rule}

This rule has two semantics depending on \val: 

\begin{enumerate}
	\item If \val is a universe, it imports a new element, assigns it to \id, and evaluates 
\crule. The resulting update set is the union of the updates generated by \crule and
a single update to add the imported element to the universe \val. 

	\item If \val is a background, it gets the default element from the background, assigns it to 
\id and evaluates \crule. The resulting update set is the updates generated by
\crule.
\end{enumerate}

In the following example, the universe \code{Agents} is extended with a new agent and 
the program of that agent is set to \code{MainProgram}:

\bExample
	\codebf{extend} Agents \codebf{with} a \codebf{do}\\
	\tab program(a) := @MainProgram
\eExample

\noindent However, the same result can be achieved by:

\bExample
	\codebf{import} a \codebf{do}\\
	\tab \codebf{par}\\
	\tab \tab Agents(a) := \codebf{true}\\
	\tab \tab program(a) := @MainProgram\\
	\tab \codebf{endpar}
\eExample


% -------------------------------- TURBO ASM
\subsection{TurboASM Rules}
\label{turboasm}

The TurboASM plugin\index{TurboASM plugin} can be loaded by the following \keyword{use} phrase:

\bExample
	\codebf{use} TurboASM
\eExample

\noindent This plugin provides the following rule forms:

\ruleform{\codebf{seq} \rulei{1} \optional{\codebf{next}} \rulei{2}}{TurboASM Plugin}\index{seq rule@\keyword{seq} rule}

Evaluates \rulei{1}, applies the generated updates in a virtual state, and evaluates \rulei{2} in that state. 
The resulting update set is a sequential composition of the updates generated by \rulei{1} and \rulei{2}.
The optional keyword \keyword{next} is meant to improve readability specially where the sequence rule is combined 
with other rule forms. 

\ruleform{\codebf{seqblock} \rulei{1} \optional{\rulei{2} \ldots \rulei{n}} \codebf{endseqblock}}{TurboASM Plugin}\index{sequence block rule}\index{seq block rule@\keyword{seqblock}|see{sequence block rule}}

Similar to the \keyword{seq} rule (above), this rule form executes the listed rules in sequence. It evaluates \rulei{1}, 
applies the generated updates in a virtual state, evaluates \rulei{2} in that state and applies the generated 
updates in a virtual state, and so on. 
The resulting update set is a sequential composition of the updates generated by \rulei{1} \ldots \rulei{n}.

\ruleform{\codebf{iterate} \crule}{TurboASM Plugin}

Repeatedly evaluates \crule, until the update set produced is either empty or inconsistent; at that point, 
the accumulated updates are computed (the resulting update set can be inconsistent if the computation of 
the last step had produced an inconsistent set of updates). 
%Evaluates \rule and if the resulting update set is not empty, applies the updates in a virtual state. 
%Repeats evaluating \rule while the resulting update set is not empty and it is not inconsistent. 

\ruleform{\codebf{while} (\val) \crule}{TurboASM Plugin}\indexrule{while}

This rule is equivalent to:

\bExample
	\codebf{iterate} \\
	\tab \codebf{if} \val \codebf{then} \crule
\eExample

\ruleform{\loc \code{<-} \crule}{TurboASM Plugin}\index{return result rule}\index{<- rule@\keyword{<-} rule|see{return result rule}}

Replaces all the occurrences of \keyword{result} in \crule with \loc and
evaluates the rule. \new{In the ASM book this is written as `$loc \leftarrow rule$'.}
In the following example, the evaluation of
\code{MainProgram} assigns the value of 5 divided by 2 (i.e., 2.5) to
\code{division}:

\bExample
	\codebf{rule} Divide(a, b) =  \\
	\tab \codebf{if} b > 0 \codebf{then}\\
	\tab \tab \codebf{result} := a / b\\
	\tab \codebf{else}\\
	\tab \tab \codebf{par}\\
	\tab \tab \tab \codebf{result} := \codebf{undef} \\
	\tab \tab \tab error := \codebf{true} ~~~\codecom{also updates the global state}\\
	\tab \tab \codebf{endpar}\\
	\\
	\codebf{rule} MainProgram = \\
	\tab division <-  Divide(5, 2)
	%\tab \codebf{seq} \\
	%\tab \tab division := 0\\
	%\tab \tab \codebf{assign} Divide(5, 2) \codebf{to} division
\eExample


\ruleform{\codebf{return} \val \codebf{in} \crule}{TurboASM Plugin}\indexrule{return}

First, \crule is evaluated; \val is then evaluated in the state obtained by provisionally 
applying the updates of \crule to the current state, and the \val is returned, while 
the updates and the provisional state itself are discarded. This rule form can be used
as an expression as it always return a value.

The above example can also be written as below:

\bExample
	\codebf{rule} Divide(a, b) =  \\
	\tab \codebf{return} d \codebf{in}\\
	\tab \tab \codebf{if} b > 0 \codebf{then}\\
	\tab \tab \tab \codebf{d} := a / b\\
	\tab \tab \codebf{else}\\
	\tab \tab \tab \codebf{par}\\
	\tab \tab \tab \tab \codebf{d} := \codebf{undef} \\
	\tab \tab \tab \tab error := \codebf{true} ~~~\codecom{has no effect}\\
	\tab \tab \tab \codebf{endpar}\\
	\\
\codebf{rule} MainProgram = \\
\tab division := Divide(5, 2) \\
\eExample

Note that this time, \code{Divide} cannot change the global state as all the updates
are discarded after \val is evaluated.

\ruleform{\codebf{local} \idi{1} \optional{, \idi{2}, \ldots , \idi{n}} \codebf{in} \crule}{TurboASM Plugin}\indexrule{local}

Evaluates \crule but discards all the updates to locations addressed by \id-s (as location names). 
In the following example, \code{newValue} will get the local value of \code{foo(5, 7)} (i.e., \code{25}) 
but the update to \code{foo(5, 7)} will be discarded afterwards.

\bExample
	\codebf{rule} LocalRule =\\
	\tab \codebf{local} foo \codebf{in}\\
	\tab \tab \codebf{seq}\\
	\tab \tab \tab foo(5, 7) := 25 \\
	\tab \tab \tab newValue := foo(5, 7) \\
\eExample


% -------------------------------- STRING
\subsection{String Background}
\label{string}

The String plugin\index{String plugin} can be loaded by the following \keyword{use} phrase:

\bExample
	\codebf{use} String
\eExample

\noindent This plugin provides the string background (\code{STRING})\index{String background} and
a small set of functions and expression forms. 

\opform{\valuei{1} \code{+} \valuei{2}}{String Plugin}\index{string concatenation}

If both values are string, this operator concatenates the given string values in to one. If one of the
values is not a string value, it tries to convert it into a string value, and then concatenates the values.
This operator is not defined on two non-string values.

With this operator, one can simply put values together to create a customized message:

\bExample
	\codebf{print} "The amount of \$" + amount + " is deposited to your account."
\eExample

\funcform{$ \mbox{\code{toString}} : \Rdom{Element} \rightarrow \Rdom{String}$}{String Plugin}\index{toString function@\code{toString} function}

A conversion function that maps any value to a String value (which can also be \keyword{undef}).

\funcform{$ \mbox{\code{strlen}} : \Rdom{String} \rightarrow \Rdom{Number}$}{String Plugin}\index{strlen function@\code{strlen} function}

Returns the length of the given String value.

\funcform{$ \mbox{\code{matches}} : \Rdom{String} \rightarrow \Rdom{String}$}{String Plugin}\index{matches function@\code{matches} function}

Returns true, if the first parameter matches the given regular expression provided by the second parameter. Otherwise false is returned. The syntax for the regular expressions follows the java language definition. For example, the function \code{matches("42", "[0-9]+")} returns true.

% -------------------------------- INPUT - OUTPUT
\subsection{Input and Output}
\label{io}

The IO plugin\index{IO plugin} can be loaded by the following \keyword{use} phrase:

\bExample
	\codebf{use} IO
\eExample

\noindent This plugin provides the following rule form and function:

\ruleform{\codebf{print} \val}{IO Plugin}\indexrule{print}

Prints out \val to the environment. Depending on the environment (engine driver) this value
can be printed on the standard output.

\funcform{$ \Rfun{input}: \Rdom{String} \rightarrow \Rdom{String}$}{IO Plugin}\index{input function@\code{input} function}

Reads a string value from the environment.
\new{Given a step and given an argument \Tfun{arg}, every evaluation of $\Rfun{input}(\Rfun{arg})$ during this step}
will result in the same value. Please refer to Section~\ref{string} for an introduction to the String
Plugin. 

The machine specified in Spec~\ref{spec:sayshello} is an extension of our This-Is-CoreASM example (see
\specref{thisiscoreasm}) that reads a name from the environment and prints
out a greeting to that name:

\begin{coreasmbox}
	\begin{spec}	
	\codebf{CoreASM} CoreASMSaysHello\\
	\\
	\codebf{use} StandardPlugins\\
	\\
	\codebf{init} InitRule\\
	\\
	\codebf{rule} InitRule = \\
	\tab \codebf{par}\\
	\tab \tab terminate := \codebf{false}\\
	\tab \tab program(self) := @MainProgram\\
	\tab \tab name := input("What is your name?")\\
	\tab \codebf{endpar}\\
	\\
	\codebf{rule} MainProgram = \\
	\tab \codebf{if} \codebf{not} terminate \codebf{then}\\
	\tab \tab \codebf{par}\\
	\tab \tab \tab \codebf{print} "This is CoreASM."\\
	\tab \tab \tab terminate := \codebf{true}\\
	\tab \tab \tab \codebf{print} "Hello " + name + "!"\\
	\tab \tab \codebf{endpar}
	\end{spec}
	\caption{CoreASM-Says-Hello example}
	\label{spec:sayshello}
\end{coreasmbox}


% -------------------------------- COLLECTION
\subsection{Collection}

The Collection plugin\index{Collection plugin} can be loaded by the following \keyword{use} phrase:

\bExample
	\codebf{use} Collection
\eExample

This plugin provides the foundation for collections (i.e., sets, lists, maps, etc.) in \CoreASM and provides some general functions 
on collections. However, each specific collection background (e.g., list or set) is provided by its corresponding plugin. 

\funcform{$\mbox{\code{foldl}} : \Rdom{Element} \times \Rdom{Function} \times \Rdom{Element} \rightarrow \Rdom{Element}$}{Collection Plugin}\index{fold-left function@\code{foldl} function}

\code{foldl(c, @func, init)} processes the collection \code{c} (e.g., a set or a list) using the binary function \code{func} and the initial value \code{init} and returns the final result.

\[ \Rfun{foldl} ([x_1,\ldots,x_n], f, i) \equiv f(x_n, f(x_{n-1}, \ldots f(x_1, init))) \ldots ) \] 
 
\funcform{$\mbox{\code{foldr}} : \Rdom{Element} \times \Rdom{Function} \times \Rdom{Element} \rightarrow \Rdom{Element}$}{Collection Plugin}\index{fold-right function@\code{foldr} function}

\code{foldr(c, @func, init)} processes the collection \code{c} (a set or a list)
using the binary function \code{func} and the initial value
\code{init} and returns the final result.

\[ \Rfun{foldr} ([x_1,\ldots,x_n], f, i) \equiv f(x_1, f(x_2, \ldots f(x_n, init))) \ldots ) \] 
 
\funcform{$\mbox{\code{fold}} : \Rdom{Element} \times \Rdom{Function} \times \Rdom{Element} \rightarrow \Rdom{Element}$}{Collection Plugin}\index{fold function@\code{fold} function}

This is the same as \code{foldr}; see above.

\funcform{$\mbox{\code{map}} : \Rdom{Element} \times \Rdom{Function} \rightarrow \Rdom{Element}$}{Collection Plugin}\index{map function@\code{map} function}

\code{map(c, @func)} applies the unary function \code{func} to all the elements of \code{c} (any collection, such as list and set) 
and returns a new collection (with the same structure as that of \code{c}).  

\[ \Rfun{map} ([x_1,\ldots,x_n], f) \equiv [ f(x_1), f(x_2), \ldots f(x_n) ] \] 

\funcform{$\mbox{\code{filter}} : \Rdom{Element} \times \Rdom{Function} \rightarrow \Rdom{Element}$}{Collection Plugin}\index{filter function@\code{filter} function}

\code{filter(c, @func)} applies the boolean unary function \code{func} to all the elements of \code{c} and returns a new collection with only those elements of \code{c} for which \code{func} returns \code{true}. 


% -------------------------------- SET
\subsection{Set Background}
\label{set}

The Set plugin\index{Set plugin} can be loaded by the following \keyword{use} phrase:

\bExample
	\codebf{use} Set
\eExample

\noindent This plugin provides the set background (\code{SET})\index{Set background} and
a a number of functions and expression forms. 

\opform{\code{\{} \optional{\valuei{1}, \ldots, \valuei{n}} \code{\}}}{Set Plugin}\index{set element}\index{\{\}@\texttt{"\{\ldots"\}}|see{set element}}\index{set enumeration}

Creates a set element that includes the listed values. The values should be basic terms (i.e., no operators) or
they should be surrounded in parentheses.

\opform{\code{\{} \id $|$ \id \codebf{in} \val \optional{\codebf{with} \guard} \code{\}}}{Set Plugin}\index{set element}\index{set comprehension}

This is the basic form set comprehension. It creates a set of all the elements in \val
which satisfy \guard. Of course, \val must be enumerable.

\opform{\code{\{} \id \codebf{is} {\em exp}\xspace $|$ \idi{1} \codebf{in} \valuei{1} 
		\optional{, \ldots, \idi{n} \codebf{in} \valuei{n}} \codebf{with} \guard \code{\}}}{Set Plugin}\index{set element}\index{set comprehension}

Creates a set element that contains all the elements of the form {\em exp} which
satisfy the \guard. In this form, {\em exp} is a function of \idi{1}\ldots \idi{n} and 
every \idi{i} is bound to an enumerable \valuei{i}. 

In the following example, \code{SetAdd} takes two sets \code{set1} and \code{set2} as input and produces a new set
by adding every element of \code{set1} to all the elements of \code{set2}:   

\bExample
	\codebf{rule} SetAdd(set1, set2) = \\
	\tab \codebf{return} a \codebf{in}\\
	\tab \tab a := \{ x \codebf{is} (x1 + x2) $|$ x1 \codebf{in} set1, x2 \codebf{in} set2 \}\\ 
\eExample

The result of evaluating \code{SetAdd(\{1, 2, 3\}, \{10, 20\})} would be:

\begin{quote} \code{ \{22.0, 23.0, 12.0, 21.0, 13.0, 11.0\}} \end{quote}

\opform{\valuei{1} bin-op \valuei{2}}{Set Plugin}

Performs a set binary operation where both \valuei{1} and \valuei{2} are sets. 
Currently, \code{subset}\indexexp{subset}, \code{union}\indexexp{union}, \code{intersect}\indexexp{intersect}, and \code{diff}\indexexp{diff} are supported.

\smallskip
Set background also provides two important rule forms which allow for parallel incremental
updates of set data structures. 

\ruleform{\codebf{add} \val \codebf{to} \loc}{Set Plugin}\indexrule{add-to-set}

If \loc is a location in the state (e.g., a function) and its value is a set, this rule 
produces an update instruction (partial update) that adds \val to \loc.

\ruleform{\codebf{remove} \val \codebf{from} \loc}{Set Plugin}\indexrule{remove-from-set}

If \loc is a location in the state (e.g., a function) and its value is a set, 
this rule produces an update instruction (partial update)
that removes \val to \loc.


% -------------------------------- LIST
\subsection{List Background}
\label{list}

The List plugin\index{List plugin} can be loaded by the following \keyword{use} phrase:

\bExample
	\codebf{use} List
\eExample

\noindent This plugin provides a list background (\code{LIST})\index{List background} and a rich set of functions and operators on lists.

\opform{\code{[} \optional{\valuei{1}, \valuei{2}, \ldots , \valuei{n}} \code{]}}{List Plugin}\index{list element}\index{[]@\code{[\ldots]}|see{list element}}

Creates a list element that includes \valuei{1} to \valuei{n} in the given order.\footnote{The old form of \code{<<x1,\ldots,xn>>} still works but it
is deprecated and may not be supported in future releases of the \CoreASM engine.}
List elements are enumerable. The index of the first element is 1.

\opform{\valuei{1} \code{+} \valuei{2}}{List Plugin}\index{list concatenation}

If both values are list, this operator concatenates the given lists in to one list. 

\funcform{$\mbox{\code{toList}} : \Rdom{Element} \rightarrow \Rdom{List}$}{List Plugin}\indexfunc{toList}

If \code{e} is an enumerable (e.g., number range, set, etc.), \code{toList(e)} will return a list that includes all the 
elements of \code{e}. If \code{e} is not ordered (e.g., a set), the order of elements in the returned list will be non-deterministic;
otherwise the elements will be in the same order.

\funcform{$\mbox{\code{flattenList}} : \Rdom{List} \rightarrow \Rdom{List}$}{List Plugin}\indexfunc{flattenList}

If \code{l} is a netsting list, \code{flattenList(l)} will return a flatten version of \code{l}.

\funcform{$\mbox{\code{head}} : \Rdom{List} \rightarrow \Rdom{Element}$}{List Plugin}\indexfunc{head}

Returns the first element of the list.

\funcform{$\mbox{\code{last}} : \Rdom{List} \rightarrow \Rdom{Element}$}{List Plugin}\indexfunc{last}

Returns the last element of the list.

\funcform{$\mbox{\code{tail}} : \Rdom{List} \rightarrow \Rdom{List}$}{List Plugin}\indexfunc{tail}

Returns all but the first element of the list.

\funcform{$\mbox{\code{cons}} : \Rdom{Element} \times \Rdom{List} \rightarrow \Rdom{List}$}{List Plugin}\indexfunc{cons}

Creates a new list with the given element as its head and given list as its tail.

\funcform{$\mbox{\code{nth}} : \Rdom{List} \times \Rdom{Number} \rightarrow \Rdom{Element}$}{List Plugin}\indexfunc{nth}

Returns the n$^{th}$ element of the list. The index of the first element is 1.

\funcform{$\mbox{\code{setnth}} : \Rdom{List} \times \Rdom{Number} \times \Rdom{Element} \rightarrow \Rdom{List}$}{List Plugin}\indexfunc{setnth}

\code{setnth(list, i, e)}, if \code{i} is a valid index for \code{list}, returns a new list in which the 
element at index \code{i} is \code{e}. 

\funcform{$\mbox{\code{take}} : \Rdom{List} \times \Rdom{Number} \rightarrow \Rdom{List}$}{List Plugin}\indexfunc{take}

\code{take(list, i)} returns the first $i$ elements of list \code{list}. 

\funcform{$\mbox{\code{drop}} : \Rdom{List} \times \Rdom{Number} \rightarrow \Rdom{List}$}{List Plugin}\indexfunc{drop}

\code{drop(list, i)} returns what is left after dropping the first $i$ elements of the list \code{list}. 

\funcform{$\mbox{\code{reverse}} : \Rdom{List} \rightarrow \Rdom{List}$}{List Plugin}\indexfunc{reverse}

Returns a list consisting of the given list's elements in reverse order. 

\funcform{$\mbox{\code{indexes}} : \Rdom{List} \times \Rdom{Element} \rightarrow \Rdom{List}$}{List Plugin}\indexfunc{indexes}

Returns a potentially empty list of the indexes of the given element in given list.

\funcform{$\mbox{\code{indices}} : \Rdom{List} \times \Rdom{Element} \rightarrow \Rdom{List}$}{List Plugin}\indexfunc{indices}

The same as \code{indexes}; see above.

\funcform{$\mbox{\code{zip}} : \Rdom{List} \times \Rdom{List} \rightarrow \Rdom{List}$}{List Plugin}\indexfunc{zip}

The function \code{zip} takes two lists and returns a list of corresponding pairs. If one input list is short, excess elements of the longer list are discarded. 

\funcform{$\mbox{\code{zipwith}} : \Rdom{List} \times \Rdom{List} \times \Rdom{Function} \rightarrow \Rdom{List}$}{List Plugin}\indexfunc{zipwith}

The function \code{zipwith} generalises zip by zipping with the function given as the last argument, instead of a tupling function. For example, \code{zipwith (l1, l2, @max)}  is applied to two lists to produce a list of corresponding maximums.

\funcform{$\mbox{\code{replicate}} : \Rdom{Element} \times \Rdom{Number} \rightarrow \Rdom{List}$}{List Plugin}\indexfunc{replicate}

The function \code{replicate(x, n)} returns a new list where the given element x is repeated n times. 

\medskip

List background also provides the following rule forms to manipulate lists:  

\ruleform{\codebf{add} \val \codebf{to} \loc}{List Plugin}\indexrule{add-to-list}

If \loc is a location in the state and its value is a list, this rule 
produces an update that adds \val to \loc. In lists order matters, so the update produced
by this rule is NOT incremental (not like the one for sets). As a result, there cannot be two parallel 
\code{add} rules operating on the same list. 

\ruleform{\codebf{remove} \val \codebf{from} \loc}{List Plugin}\indexrule{remove-from-list}

If \loc is a location in the state and its value is a list, 
this rule produces an update that removes the first occurrence of \val from \loc.
As for \code{add}, this rule is also NOT incremental (not like the one for sets) and there 
cannot be two parallel \code{remove} rules operating on the same list.

\ruleform{\codebf{shift left} \val \codebf{into} \loc}{List Plugin}\indexrule{shift-left-list}

If \loc is a location in the state and \val is a list, it removes the first element
of the list and puts it in the given location (shifiting the list to left).

\ruleform{\codebf{shift right} \val \codebf{into} \loc}{List Plugin}\indexrule{shift-right-list}

If \loc is a location in the state and \val is a list, it removes the last element
of the list and puts it in the given location (shifiting the list to right).

\smallskip
In the following example, \code{SortSet} sorts elements of a given set into a 
list:  
  
\bExample
	\codebf{rule} SortSet(set) = \\
	\tab \codebf{seq} \\
	\tab \tab \codebf{par}\\
	\tab \tab \tab \codebf{result} := []\\
	\tab \tab \tab tempSet := set\\
	\tab \tab \codebf{endpar}\\
	\tab \codebf{next}\\
	\tab \tab \codebf{while} ($|$ tempSet $|$ > 0) \\
	\tab \tab \tab \codebf{choose} x \codebf{in} tempSet \codebf{with} (\codebf{forall} y \codebf{in} tempSet \codebf{holds} x <= y) \codebf{do}\\ 
	\tab \tab \tab \tab \codebf{par}\\
	\tab \tab \tab \tab \tab \codebf{remove} x \codebf{from} tempSet\\
	\tab \tab \tab \tab \tab \codebf{add} x \codebf{to} result\\
	\tab \tab \tab \tab \codebf{endpar}
\eExample
 

% -------------------------------- QUEUE
\subsection{Queue}
\label{queue}

The Queue plugin\index{Queue plugin} can be loaded by the following \keyword{use} phrase:

\bExample
	\codebf{use} Queue
\eExample

\noindent This plugin provides the following queue\index{queue} operations (rule forms) on lists:

\ruleform{\codebf{enqueue} \val \codebf{into} \loc}{Queue Plugin}\indexrule{enqueue}

If \loc is a location in the state and its value is a queue (i.e., a list), it adds \val to 
the end of the queue.

\ruleform{\codebf{dequeue} \loci{v} \codebf{from} \loci{q}}{Queue Plugin}\indexrule{dequeue}

If \loci{q} is a location in the state and its value is a queue (i.e., a list), it removes 
the first element of this queue and assigns it as the value of the location \loci{v}. 


% -------------------------------- STACK
\subsection{Stack}
\label{stack}

The Stack plugin\index{Stack plugin} can be loaded by the following \keyword{use} phrase:

\bExample
	\codebf{use} Stack
\eExample

\noindent This plugin provides the following stack\index{stack} operations and functions on lists:

\ruleform{\codebf{push} \val \codebf{into} \loc}{Stack Plugin}\indexrule{push}

If \loc is a location in the state and its value is a stack (i.e., a list), it pushes \val to 
the front of the stack.

\ruleform{\codebf{pop} \loci{v} \codebf{from} \loci{s}}{Stack Plugin}\indexrule{pop}

If \loci{s} is a location in the state and its value is a stack (i.e., a list), it removes
the first element of the stack (top of the stack) and assigns it as the value of \loci{v}.

\funcform{$\mbox{\code{peek}} : \Rdom{List} \rightarrow \Rdom{Element}$}{Stack Plugin}\indexfunc{peek}

Returns the top of the stack (first element of the list) without changing the stack.


% -------------------------------- MAP
\subsection{Map Background}
\label{map}

The Map plugin\index{Map plugin} can be loaded by the following \keyword{use} phrase:

\bExample
	\codebf{use} Map
\eExample

\noindent This plugin provides a map background (\code{MAP})\index{Map background}.

\opform{\code{\{} \code{->} \code{\}}}{Map Plugin}\index{map element}\index{[]@\code{[\ldots]}|see{map element}}

Creates an empty map.

\opform{\code{\{} \valuei{k1} \code{->} \valuei{v1}, \optional{\valuei{k2} \code{->} \valuei{v2}, \ldots , \valuei{kn} \code{->} \valuei{vn}} \code{\}}}{Map Plugin}\index{map element}\index{[]@\code{[\ldots]}|see{map element}}

Creates a map with the given key-value pairs. Map elements are enumerable; every map can be viewed as a set of pairs  
which are represented by lists of size 2.

\funcform{$\mbox{\code{toMap}} : \Rdom{Element} \rightarrow \Rdom{Map}$}{Map Plugin}\indexfunc{toMap}

If \code{e} is an enumerable (e.g., a set) consisting of pairs of elements (lists of size two) 
of the form $[k_i, v_i]$ such that $\forall [k_i, v_i] \not\exists [k_j, v_j] ~~k_i = k_j \wedge v_i \neq v_j$, 
\code{toMap(e)} returns a map element representing a mapping of $k_i$s to $v_i$s;
otherwise, it returns \keyword{undef}. For example, the following two expressions create equal maps:

\bExample
	toMap(\{[1, "John"], [2, "Mary"]\})\\
	\{ 1 -> "John", 2 -> "Mary"\}
\eExample

\funcform{$\mbox{\code{mapToPairs}} : \Rdom{Map} \rightarrow \Rdom{Set}$}{Map Plugin}\indexfunc{mapToPairs}

Returns a set of pairs of the form $(key, value)$ from the given map elements. The pairs are list elements of
size two. For example, the following two expressions are equal:

\bExample
	mapToPairs( \{1 -> "John", 2 -> "Mary"\} )\\
	\{[1, "John"], [2, "Mary"]\}
\eExample


Map background also provides the following rule forms to manipulate maps:  

\ruleform{\codebf{add} \val \codebf{to} \loc}{Map Plugin}\indexrule{add-to-map}

If \loc is a location in the state, its value is a map, and \val is a map, this rule 
produces an update that copied all of the mappings from \val to \loc. These mappings 
will replace any mappings that \loc had for any of the keys shared with \val. 
In the current version of Map plugin, the update produced
by this rule is NOT incremental (not like the one for sets). As a result, there cannot be two parallel 
\code{add} rules operating on the same map. 

\ruleform{\codebf{remove} \val \codebf{from} \loc}{Map Plugin}\indexrule{remove-from-map}

If \loc is a location in the state and its value is a map, this rule produces an 
update that removes \val from \loc according to the following:
\begin{enumerate}
	\item if \val is a map, this rule removes all the exact mappings 
		of \val from \loc;
	\item if \val is not a map but an enumerable, this rule removes all the 
		mappings for the elements of \val (as keys) from \loc;
	\item if \val is neither a map nor an enumerable, this rule removes the mapping
		for \val (as a key) from \loc if present.
\end{enumerate} 
In the current version of Map plugin, the update produced
by this rule is NOT incremental (not like the one for sets). As a result, there cannot be two parallel 
\code{remove} rules operating on the same map. 

% -------------------------------- SIGNATURE
\subsection{Signature Plugin}
\label{signature}

The Signature plugin\index{Signature plugin} can be loaded by the following \keyword{use} phrase:

\bExample
	\codebf{use} Signature
\eExample

\noindent The signature plugin extends the header section of \CoreASM
specifications (see Figure~\ref{fig:specStructure}) to add support for definition of functions,
universes, and custom data types and also extends the engine to support for
certain forms of type checking. This plugin is still under development. The current
version includes the following features: 

\begin{itemize}
	\item Definition of universes through the following syntax (with optional initial elements):
		\begin{quote} \code{ \codebf{universe} \id \optional{= \{ \idi{1}, \ldots, \idi{n} \}} } \end{quote}\index{universe@\keyword{universe}}

	\item Definition of enumeration backgrounds through the following syntax:
		\begin{quote} \code{ \codebf{enum} \idi{e} = \{ \idi{1} \optional{, \ldots, \idi{n}} \} } \end{quote}\index{enumeration backgrounds}\index{enum@\keyword{enum}|see{enumeration background}}
		For example, the following line defines a new enumeration background of four elements:

		\bExample
			\code{ \codebf{enum} PRODUCT = \{ Soda, Juice, Sandwich, Candy \} }
		\eExample
	
		The elements are in fact defined as constant functions that hold values of the background \code{PRODUCT}. 

	\item Definition of functions through the following syntax:
		\begin{quote} \code{ \codebf{function} \idi{f} : \optional{\idi{u1} * \ldots * \idi{un}} -> \idi{r}} \end{quote}\index{function@\keyword{function}}\index{controlled@\keyword{controlled}} 
		As an example, the following signature defines a function named \code{priceTable} that maps pairs of string values to numbers:
		
		\bExample
			\code{ \codebf{function} priceTable : STRING * STRING -> NUMBER }	
		\eExample

	\item Definition of derived functions through the following syntax:
		\begin{quote} \code{ \codebf{derived} \idi{f} \optional{( \idi{1}, \ldots, \idi{n})} = {\em expression} } \end{quote}\index{derived function@\keyword{derived}} 
		As an example, the following declaration defines a derived function $f(x, y) = x^2 + y^2$: 
		
		\bExample
			\code{ \codebf{derived} f(x, y) = x\^{}2 + y\^{}2  }	
		\eExample

\end{itemize}

\medskip

Depending on the properties of the engine (see the Options Plugin, Section~\ref{options}) the Signature plugin can use the signature information 
to perform the following checks:
\begin{itemize}

\item {\bf Type checking on assignments}: if the
``\code{Signature.TypeChecking}'' property is set to ``\code{warning}'',
``\code{strict}'' or ``\code{on}'', before the updates are applied to the
state, the Signature Plugin checks the types of arguments and assigned values
against the defined signatures and issues a warning (in case of
``\code{warning}'') or stops the execution of the engine with an error (in case
of ``\code{strict}'' or ``\code{on}'').     


\item {\bf Unknown identifiers:} if the ``\code{Signature.NoUndefinedId}''
property is set to ``\code{warning}'', ``\code{strict}'' or ``\code{on}'', the
Signature Plugin issues a warning (in case of ``\code{warning}'') or stops the
execution of the engine with an error (in case of ``\code{strict}'' or
``\code{on}'') if a function name is used and its signature is not defined in
the header of the specification. This feature helps in identifying typos
in the specification. 

\end{itemize}

\section{Additional Plugins}
\label{additional}

The plugins listed in this section are currently not part of any plugin packages.

% -------------------------------- MODULARITY
\subsection{Modularity}
\label{Modularity}

The Modularity plugin\index{Modularity plugin} can be loaded by the following \keyword{use} phrase:

\bExample
	\codebf{use} Modularity
\eExample

This plugin allows one to break the specification into separate files or {\em modules}. As its current version, 
the functionality provided is limited to introducing an \codebf{include} keyword that would load another file 
into the current specification. 

\begin{quote} \code{ \codebf{include} {\em filename}} \end{quote}\index{include@\keyword{include}}

Included files can themselves have other \codebf{include} clauses to further break down 
the specification.

% -------------------------------- OPTIONS
\subsection{Options}
\label{options}

The Options plugin\index{Options plugin} can be loaded by the following \keyword{use} phrase:

\bExample
	\codebf{use} Options
\eExample

\noindent The Options plugin extends the header section of \CoreASM
specifications (see Figure~\ref{fig:specStructure}) to provide the following syntax i
to set values of engine properties: 

\begin{quote} \code{ \codebf{option} {\em property} {\em value} } \end{quote}\index{option@\keyword{option}}

Other plugins (such as the Signature Plugin, see Section~\ref{signature}) can use these options to 
customize their behavior.

% -------------------------------- SCHEDULING
\subsection{Scheduling Policies}
\label{scheduling}

The Scheduling Policies plugin\index{Scheduling Policies plugin} can be loaded by the following \keyword{use} phrase:

\bExample
	\codebf{use} SchedulingPolicies
\eExample

This plugin provides alternative scheduling policies for simulation of multi-agent specifications. 
For any specification (for any run), only one scheduling policy can be defined, using the following option:

\begin{quote} \code{ \codebf{option} \code{SchedulingPolicies.policy} {\em policyname} } \end{quote}

Currently, there are two scheduling policies provided by this plugin:
\begin{itemize}
  \item \codebf{allfirst} Tries executing all the agents in every computation step. If this fails at any step,
	the policy falls back to the engine's default scheduling policy.

  \item \codebf{onebyone} Executes only one agent in every step. It tries to be {\em fair} by not executing an 
	agent more than once unless all other agents have been given a chance to execute.

\end{itemize}

The following rules are also provided by this plugin to control the execution of agents during a simulation.

\ruleform{\codebf{suspend} \val}{SchedulingPolicies Plugin}\indexrule{suspend}

If \val is an agent, this rule {\em suspends} the execution of that agent from the next computation step.
The suspended agents will not be chosen by the engine for execution.  

\ruleform{\codebf{resume} \val}{SchedulingPolicies Plugin}\indexrule{resume}

If \val is an agent which has been suspended, this rule {\em resumes} the execution of that agent from the next computation step;
i.e., the agent will be available for execution from the next step.

\ruleform{\codebf{terminate} \val}{SchedulingPolicies Plugin}\indexrule{terminate}

If \val is an agent, it will no longer be available for scheduling for the rest of the current run of the machine. 

\ruleform{\codebf{shutdown}}{SchedulingPolicies Plugin}\indexrule{shutdown}

Clears the \code{Agents} universe, such that there will be no agent available to contribute to the 
next computation step.  Depending on the parameters of the run, this can stop the execution of the engine.

% -------------------------------- TIME
\subsection{Time}
\label{time}

The Time plugin\index{Time plugin} can be loaded by the following \keyword{use} phrase:

\bExample
	\codebf{use} Time
\eExample

\noindent This plugin provides the following monitored function: 

\funcform{$ \mbox{\code{now}} : \rightarrow \Rdom{Number}$}{Time Plugin}\index{now function@\keyword{now} function}

Returns a value representing the current time of the system. \new{Of course, given a step, the value of now is fixed.}

\funcform{$ \mbox{\code{stepcount}} : \rightarrow \Rdom{Number}$}{Time Plugin}\index{stepcount function@\keyword{stepcount} function}

Returns the number of computation steps performed so far by the engine excluding the current step.

% -------------------------------- DEBUG INFO
\subsection{DebugInfo} 
\label{debugInfo}

DebugInfo plugin is a \CoreASM plugin to maintain logging information for debugging purposes and it 
can be loaded by the following \keyword{use} phrase:

\bExample
	\codebf{use} DebugInfo
\eExample

The plugin adds the following rule to the \CoreASM language:

\ruleform{\codebf{debuginfo} \id\ \val}{DebugInfo Plugin}\indexrule{debuginfo}

\noindent which, upon evaluation, adds the string representation of the given \val to the logging channel identified by the given \id. 

The set of active channels are to be defined as a space-separated list of channel ids, set as the value
\code{DebugInfo.activeChannels} engine property. This can be done either through the Options plugin or 
by setting the values directly from the engine driver (e.g., \Carma). For example, using the Options plugin one can add the following
line to a spec to turn the logging on for channels {\em warning} and {\em error}:

\begin{coreasm}
  \clkeyword{option} DebugInfo.activeChannels "warning, error"
\end{coreasm}

\noindent In order to turn all channels on, one can use the special channel id {\em ALL}:

\begin{coreasm}
 \clkeyword{option} DebugInfo.activeChannels ALL \clcomment{// or "ALL"}
\end{coreasm}

Since this rule is only used for debugging purposes, 
the evaluation of \codebf{debuginfo} results in an empty update set and a print out of the debugging information (if the corresponding 
channel is active) to the standard output, whether or not the updates of the enclosing rule block is discarded by the engine or not. 
Applications of the engine can set redirect the output of 
this plugin using the plugin's service interface (see \code{org.coreasm.engine.plugin.Plugin\#getPluginInterface()}). 

\subsubsection*{Example}

\begin{coreasm}
\clkeyword{CoreASM}  DebugInfoExample\\
\\
\clkeyword{use}  Standard\\
\clkeyword{use}  DebugInfo\\
\clkeyword{use}  Options\\
\\
\clkeyword{option}  DebugInfo.activeChannels ALL\clcomment{\\
//option DebugInfo.activeChannels "ch1 ch2"\\
//option DebugInfo.activeChannels "ch1,ch2"\\
//option DebugInfo.activeChannels ch1\\
//option DebugInfo.activeChannels NONE\\
}\\  
\clkeyword{init}  R1\\
\\
\clkeyword{rule}  R1 = \{\\
\cltb  \clkeyword{if}  mode =  \clkeyword{undef}   \clkeyword{then}  \{\\
\cltb \cltb  \clkeyword{debuginfo}  ch1  \clstr{"initializing."} \\
\cltb \cltb mode:=  \clstr{"counting"} \\
\cltb \cltb counter:= 0\\
\cltb \}  \clkeyword{else}  \{\\
\cltb \cltb  \clkeyword{debuginfo}  ch2  \clstr{"counting."} \\
\cltb \cltb counter:= counter + 1\\
\cltb \}\\
\}\\
\\

\end{coreasm}


% -------------------------------- MATH
\subsection{Math}
\label{math}

The Math plugin\index{Math plugin} can be loaded by the following \keyword{use} phrase:

\bExample
	\codebf{use} Math
\eExample

\noindent Math Plugin extends the \CoreASM engine to provide some basic mathematical functions. Most of these functions are equivalent of their Java counterparts in \code{java.lang.Math}. For such functions, the following descriptions are basically taken from the {\em Java 2 Platform Standard Edition 5.0 API Specification}. 

\subsubsection{Constants}

\begin{itemize}
\item {\bf \code{MathE}} \\ The value that is closer than any other to $e$, the base of the natural logarithms.
\item {\bf \code{MathPI}} \\   The value that is closer than any other to $\pi$, the ratio of the circumference of a 
 circle to its diameter. 
\end{itemize}

\subsubsection{Basic Derived Functions}

%\begin{tabular}{l|l}
\begin{itemize}
\item {\bf \code{abs(v)} }   Returns the absolute value of $v$.
\item {\bf \code{acos(v)} }   Returns the arc cosine of an angle, in the range of $0$ through $\pi$.
\item {\bf \code{asin(v)} }   Returns the arc sine of an angle, in the range of $-\pi/2$ through $\pi/2$.
\item {\bf \code{atan(v)} }   Returns the arc tangent of an angle, in the range of $-\pi/2$ through $\pi/2$.
\item {\bf \code{atan2(x, y)} }  Converts rectangular coordinates $(x, y)$ to polar $(r, \theta)$ and returns $\theta$.
\item {\bf \code{cuberoot(v)} }   Returns the cube root of $v$.
\item {\bf \code{cbrt(v)} }   Returns the cube root of $v$.
\item {\bf \code{ceil(v)} }   Returns the smallest (closest to negative infinity) value that is greater than or equal to the argument and is equal to a mathematical integer.
\item {\bf \code{cos(v)} }   Returns the trigonometric cosine of an angle.
\item {\bf \code{cosh(v)} }   Returns the hyperbolic cosine of $v$.
\item {\bf \code{exp(v)} }   Returns Euler's number $e$ raised to the power of $v$.
\item {\bf \code{expm1(v)} }   Returns $e^v -1$.
\item {\bf \code{floor(v)} }   Returns the largest (closest to positive infinity) value that is less than or equal to the argument and is equal to a mathematical integer.
\item {\bf \code{hypot(x, y)} }   Returns $\sqrt{x^2 + y^2}$ without intermediate overflow or underflow.
\item {\bf \code{IEEEremainder(v1, v2)} }   Computes the remainder operation on two arguments as prescribed by the IEEE 754 standard.
\item {\bf \code{log(v)} }   Returns the natural logarithm (base $e$) of $v$.
\item {\bf \code{log10(v)} }   Returns the base $10$ logarithm of $v$.
\item {\bf \code{log1p(v)} }   Returns the natural logarithm of the sum of the argument and 1; i.e., $ln(v + 1)$.
\item {\bf \code{max(v1, v2)} }   Returns the greater of two values.
%\item {\bf \code{max(\{v_1, \ldots, v_n\})} }  Returns the maximum value in a collection of numbers. 
%												 If there is one non-number in the collection, it returns \Rundef.
\item {\bf \code{min(v1, v2)} }   Returns the smaller of two values.
%\item {\bf \code{min(\{v_1, \ldots, v_n\})} }  Returns the minimum value in a collection of numbers.
%												 If there is one non-number in the collection, it returns \Rundef.
\item {\bf \code{pow(x, y)} }   Returns the value of the first argument raised to the power of the second argument.
\item {\bf \code{random()} }   Returns a random value with a positive sign, greater than or equal to $0.0$ and less than $1.0$.
\item {\bf \code{round(v)} }   Returns the closest mathematical integer to the argument.
\item {\bf \code{signum(v)} }  Returns zero if the argument is zero, $1.0$ if the argument is greater than zero, $-1.0$ if the argument is less than zero.
\item {\bf \code{sin(v)} }   Returns the trigonometric sine of an angle.
\item {\bf \code{sinh(v)} }   Returns the hyperbolic sine of $v$.
\item {\bf \code{sqrt(v)} }   Returns the correctly rounded positive square root of $v$; i.e., $\sqrt{v}$.
\item {\bf \code{tan(v)} }   Returns the trigonometric tangent of an angle.
\item {\bf \code{tanh(v)} }   Returns the hyperbolic tangent of $v$.
\item {\bf \code{toDegrees(v)} }   Converts an angle measured in radians to an approximately equivalent angle measured in degrees.
\item {\bf \code{toRadians(v)} }   Converts an angle measured in degrees to an approximately equivalent angle measured in radians.
%\item {\bf \code{sum(\{v_1, \ldots, v_n\})} }   This function returns the sum of a collection of numbers. If there is one non-number in the collection, it returns \Rundef.
\end{itemize}
%\end{tabular}

\subsubsection{Special Derived Functions}

\begin{itemize}

\item {\bf \code{powerset(set)} }  Computes the powerset of the given set. 

\item {\bf \code{max(\{v1,\ldots,vn\})} }  Returns the maximum value in a collection of numbers. 
	If there is one non-number in the collection, it returns \Rundef.

%\item {\bf \code{max(\{v1,\ldots, vn\}, @f)} }  Returns the maximum value in a collection of numbers, 
%	after applying function {\bf \code{f}} to the values in the collection. If there is one non-number 
%	in the collection, it returns \Rundef.

\item {\bf \code{min(\{v1,\ldots,vn\})} }  Returns the minimum value in a collection of numbers.
	If there is one non-number in the collection, it returns \Rundef.

%\item {\bf \code{min(\{v1,\ldots, vn\}, @f)} }  Returns the minimum value in a collection of numbers, 
%	after applying function {\bf \code{f}} to the values in the collection. If there is one non-number 
%	in the collection, it returns \Rundef.

\item {\bf \code{sum(\{v1,\ldots,vn\})} }   This function returns the sum of a collection of numbers. 
	If there is one non-number in the collection, it returns \Rundef.

\item {\bf \code{sum(\{v1,\ldots,vn\}, @f)} }   This function returns the sum of a collection of numbers, 
	after applying function {\bf \code{f}} to the values in the collection. If there is one non-number 
	in the collection, it returns \Rundef.

\item {\bf \code{powerset(\{e1,\ldots,en\})} } This function returns the powerset of the given set of elements.

\end{itemize}

\subsubsection{An Example}

\begin{coreasmbox}
	\begin{spec}
	\codebf{CoreASM} MathPluginExample\\
	\\
	\codebf{use} StandardPlugins\\
	\codebf{use} MathPlugin\\
	\\
	\codebf{init} Init\\
	\\
	\codebf{rule} Init = \\
	\codebf{par}\\
	\tab program(self) := @Main\\
	\tab a(1) := 5\\
	\tab a(2) := 10\\
	\tab a(100) := 500\\
	\codebf{endpar}\\
	\\
	\codebf{rule} Main = \\
	\tab \codebf{let} e = MathE \codebf{in}\\
	\tab \codebf{par}\\
	\tab \tab \codebf{print} "'e' = " + e\\
	\tab \tab \codebf{print}print "log(e) = " + log(e)\\
	\tab \tab \codebf{print} "sin(30) = " + round( sin( toRadians(30) ) * 10 ) / 10\\
	\tab \tab \codebf{print} "asin(0.5) = " + round( toDegrees( asin(0.5) ) )\\
	\tab \tab \codebf{print} "min(51, 43) = " + min(51, 43)\\
	\tab \tab \codebf{print} "sum( \{1, 2, 100\} ) = " + sum(\{1, 2, 100\})\\
	\tab \tab \codebf{print} "sum( \{1, 2, 100\}, @a ) = " + sum(\{1, 2, 100\}, @a)\\
    %\tab \tab \codebf{print} "\{2, 3\} is in P(\{1, 2, 3\}) = " + (\{2, 3\} memberof powerset(\{1,2,3\})) \\
    \tab \tab \codebf{choose} x \codebf{in} powerset(\{1, 2, 3, 4\}) \codebf{do} \\
    \tab \tab \tab \codebf{if} x \codebf{memberof} powerset(\{1, 2, 3\}) \codebf{then} \\
    \tab \tab \tab \tab \codebf{print} x + " is a member of powerset(\{1, 2, 3\})"\\
    \tab \tab \tab \codebf{else}\\ 
    \tab \tab \tab \tab \codebf{print} x + " is not a member of powerset(\{1, 2, 3\})"\\
	\tab \codebf{endpar}\\
	\end{spec}
	\caption{Using Math Plugin}
	\label{spec:mathexample}
\end{coreasmbox}

As an example, the output of the \CoreASM Spec~\ref{spec:mathexample} would be the following:

\definecolor{shellgray}{rgb}{0.2,0.2,0.2}
\definecolor{black}{rgb}{0,0,0}
\newenvironment{shell}
	{\noindent \color{shellgray}\vspace{0.2cm} \begin{minipage}{0.9\textwidth} \vspace{0.2cm}}
	{\vspace{0.2cm} \end{minipage} \vspace{0.2cm} \color{black}}

\begin{shell}
\begin{verbatim}
sum( {1, 2, 100} ) = 103
min(51, 43) = 43
asin(0.5) = 30
powerset({1, 2, 3}) = {{}, {3}, {2}, {3, 2}, {1}, {3, 1}, {2, 1}, {3, 2, 1}}
{2, 3} memberof powerset({1, 2, 3} = true
log(e) = 1
sum( {1, 2, 100}, @a ) = 515
'e' = 2.718281828459045
{2, 1, 4} is not a member of powerset({1, 2, 3})
sin(30) = 0.5
\end{verbatim}
\end{shell}

\printindex

\end{document}

